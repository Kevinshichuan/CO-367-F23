\documentclass[12pt]{article}
\usepackage{comment}
\usepackage{enumerate}
\usepackage{graphics,graphicx}
\usepackage{ifthen, xspace}
\usepackage{amssymb,amsmath,amsthm}
%\usepackage{hyperref}

%\setcounter{section}{-1}







\newtheorem*{solution}{Solution}
\newtheorem*{marking}{Marking Scheme}
\newtheorem*{comments}{Comments}
\newenvironment{packed_itemize}{
\begin{itemize}
}{\end{itemize}}
%\excludecomment{solution}   % comment out to include solutions
%\excludecomment{comments}   % comment out to include comments
%\excludecomment{marking}   % comment out to include marking scheme

%
\textwidth = 6.5 in
\textheight = 9 in
\oddsidemargin = 0.0 in
\evensidemargin = 0.0 in
\topmargin = 0.0 in
\headheight = 0.0 in
\headsep = 0.0 in
\parskip = 0.2in
\parindent = 0.0in

%\input{../notation}



%%%%%%%%%%%%%%%%%%%%%%%%%%%%%%%%%%%%%%%%%%%%%%%%%%%%%%%%%%%%%%%%%%%%%%%%%%%%%%%%%%%%%%%%%
%%%%%%%%%%%%%%%%%%%%%%%%%%%%%%%%%%%%%%%%%%%%%%%%%%%%%%%%%%%%%%%%%%%%%%%%%%%%%%%%%%%%%%%%%
%%%%%%%%%%%%%%%%%%%%%%%%%%%%%%%%%%%%%%%%%%%%%%%Henry file aliases
\usepackage{cite}
\usepackage{subfigure}
\usepackage{graphicx, color, graphpap}
\usepackage{amsmath,amssymb,amsthm}
\usepackage{ifthen}
\usepackage{mathrsfs}
\usepackage{mathtools}
\usepackage{enumitem}
\usepackage{physics}
\usepackage{pstricks}
\usepackage{lineno}
%\usepackage{dsfont}
\usepackage{float}
\usepackage{algorithmic}
\usepackage{fancyhdr}
\usepackage[us,12hr]{datetime} 
\usepackage{exscale}
\usepackage{tabularx}
\usepackage{latexsym}
\usepackage{fullpage}       
\usepackage{float}
\usepackage{verbatim}
\usepackage{multirow}
\usepackage{overpic}
%\usepackage[T1]{fontenc}
\usepackage{comment} % writing comments
\usepackage{lscape}
\usepackage{booktabs}
\usepackage{makeidx}
\usepackage{algorithm,algorithmic}
%\usepackage[mathscr]{eucal} % calligraph letters y
\usepackage{mathrsfs}


\numberwithin{equation}{section}  % enables section numbering
\numberwithin{table}{section}
\numberwithin{algorithm}{section}
\makeindex

\usepackage{csvsimple}

\usepackage[pagebackref=true]{hyperref}
% with basic options
% N.B. pagebackref=true provides links back from the References to the 
%body text. This can cause trouble for printing.
\hypersetup{
plainpages=false,       % needed if Roman numbers in frontpages
% pdfpagelabels=true,     % adds page number as label in Acrobat's page
%%count
% bookmarks=true,         % show bookmarks bar?
unicode=false,          % non-Latin characters in Acrobat bookmarks
pdftoolbar=true,        % show Acrobat toolbar?
pdfmenubar=true,        % show Acrobat menu?
pdffitwindow=false,     % window fit to page when opened
pdfstartview={FitH},    % fits the width of the page to the window
pdftitle={F23: CO367: Assign 0
},    % title: 
%CHANGE THIS TEXT!
%    pdfauthor={Author},    % author: CHANGE THIS TEXT! and uncomment 
%%this line
%    pdfsubject={Subject},  % subject: CHANGE THIS TEXT! and uncomment 
%%this line
%    pdfkeywords={keyword1} {key2} {key3}, % list of keywords, and 
%%uncomment this line if desired
pdfnewwindow=true,      % links in new window
colorlinks=true,        % false: boxed links; true: colored links
linkcolor=blue,         % color of internal links
citecolor=green,        % color of links to bibliography
filecolor=magenta,      % color of file links
urlcolor=cyan           % color of external links
}
\newcommand*\mathinhead[2]{\texorpdfstring{$\boldsymbol{#1}$}{#2}}
\usepackage[noabbrev]{cleveref}
%\usepackage{refcheck}
%%% Infrastructure begin: This is to make refcheck work with cleveref. 
%%%Comment and uncomment together with refcheck.
\makeatletter
%\newcommand{\refcheckize}[1]{%
%	\expandafter\let\csname @@\string#1\endcsname#1%
%	\expandafter\DeclareRobustCommand\csname relax\string#1\endcsname[1]{%
%		\csname @@\string#1\endcsname{##1}\wrtusdrf{##1}}%
%	\expandafter\let\expandafter#1\csname relax\string#1\endcsname
%}
\makeatother
%%%

%%% Now we add the reference commands we want refcheck to be aware of
%\refcheckize{\cref}
%\refcheckize{\Cref}
%%% Infrastructure end.
\DeclareMathOperator{\proj}{proj}
\DeclareMathOperator{\spn}{span}
\DeclareMathOperator{\argmin}{argmin}
\DeclareMathOperator{\argmax}{argmax}

\usepackage{mathtools}
\DeclarePairedDelimiter{\ceil}{\lceil}{\rceil}

\usepackage{refcount} % for counting eqtn number in enumerate enviornment


\def\Rnbyn{\mathbb{R}^{n\times n}}
\def\R{\mathbb{R}}
\def\N{\mathbb{N}}
\def\Sc{\mathbb{S}}
\def\Sn{\Sc^n}
\def\Snt{\Sc^{n^2+1}}
\def\Sntm{\Sc^{(n-1)^2+1}}
\def\Sr{\Sc^r}
\def\Snp{\Sc_+^n}
\def\Sp{\Sc_+}
\def\Srp{\Sc_+^r}
\def\Snr{\Sc^{n-r}}
\def\Snrp{\Sc_+^{n-r}}
\def\Snpp{\Sc_{++}^n}
\def\Srpp{\Sc_{++}^r}
\def\Snrpp{\Sc_{++}^{n-r}}
%\def\real{\mathbb{R}}
\def\H{\mathbb{H}}
\def\D{\mathbb{D}}
\def\Rk{\mathbb{R}^k}
\def\Rl{\mathbb{R}^l}
\def\Rd{\mathbb{R}^d}
\def\Rn{\mathbb{R}^n}
\def\Rnt{\mathbb{R}^{n^2}}
\def\Rntp{\mathbb{R}_+^{n^2}}
\def\Rntm{\mathbb{R}^{2n-1}}
\def\Rr{\mathbb{R}^r}
\def\Rq{\mathbb{R}^q}
\def\Rmm{\mathbb{R}^{m+1}}
\def\Rnt{\mathbb{R}^{n^2}}
\def\Rtnm{\mathbb{R}^{t(n)-n}}
\def\Rnp{\mathbb{R}_+^n}
\def\C{\mathbb{C}}
\def\Cn{\mathbb{C}^n}
\def\Cm{\mathbb{C}^m}
\def\Ctnm{\mathbb{C}^{t(n)-n}}

\def\bY{\overline{Y\strut}}
%\def\hY{\widehat{Y\strut}}
\def\hY{\widehat{Y}}
%\def\hV{\widehat{V\strut}}
\def\hV{\widehat{V}}
\def\Fh{F_h}
\def\FFh{\mathcal{F}_h}
\def\Fnh{F_{nh}}
\def\FFnh{\mathcal{F}_{nh}}
\def\Ah{\mathcal{A}_h}
\def\Anh{\mathcal{A}_{nh}}
\def\bh{b_h}
\def\bnh{b_{nh}}

\def\normF#1{\|#1\|_F}
\def\normt#1{\|#1\|_2}

%\def\eqref#1{{\normalfont(\ref{#1})}}
\def\GN{{\bf GN\,}}
\def\GE{\mbox{\boldmath$GE$}\,}
\def\PD{\mbox{\boldmath$PD$}\,}
\def\PDp{\mbox{\boldmath$PD$}}
\def\PSD{\mbox{\boldmath$PSD$}\,}
\def\PSDp{\mbox{\boldmath$PSD$}}


\newenvironment{note}{\begin{quote}\small\sf NJH 
$\diamondsuit$~}{\end{quote}}

% Henry's additional definition
\newenvironment{noteH}{\begin{quote}\small\sf \color{blue} HenryW 
$\clubsuit$~}{\end{quote}}
\newenvironment{noteHS}{\begin{quote}\small\sf \color{purple} Haesol 
$\spadesuit$~}{\end{quote}}
\newenvironment{noteTW}{\begin{quote}\small\sf \color{magenta} Tyler 
$\spadesuit$~}{\end{quote}}

%%%%%%%%%%%%%%%%%%%%%%%%%%%%%%%%%
% All refs in roman.
%\def\eqref#1{{\normalfont(\ref{#1})}}
%%%%%%%%%%%%%%%%%%%%%%%%%%%%%%%%%

%\newtheorem{nmbrs}{Numbering}[section]
%\newtheorem{defi}[nmbrs]{Definition}
\newtheorem{theorem}{Theorem}[section]
\newtheorem{defi}[theorem]{Definition}
\newtheorem{definition}[theorem]{Definition}
\newtheorem{example}[theorem]{Example}
\newtheorem{assump}[theorem]{Assumption}
\newtheorem{prop}[theorem]{Proposition}
\newtheorem{prob}[theorem]{Problem}
\newtheorem{ques}[theorem]{Question}
\newtheorem{lem}[theorem]{Lemma}
\newtheorem{thm}[theorem]{Theorem}
\newtheorem{cor}[theorem]{Corollary}
\newtheorem{corollary}[theorem]{Corollary}
\newtheorem{rem}[theorem]{Remark}
\newtheorem{properties}[theorem]{Properties}
%\newtheorem{table}[theorem]{table}
\newtheorem{remark}[theorem]{Remark}
\newtheorem{conj}[theorem]{Conjecture}
\newtheorem{alg}[theorem]{Algorithm}
\newtheorem{ex}[theorem]{Exercise}
\newtheorem{problem}[theorem]{Problem}
\newtheorem{question}[theorem]{Questin}
\newtheorem{conjecture}[theorem]{Conjecture}
\newtheorem{result}[theorem]{Result}
\newtheorem{claim}[theorem]{Claim}
\newtheorem{lemma}[theorem]{Lemma}
\newtheorem{coroll}[theorem]{Corollary}
\newtheorem{theo}[theorem]{Theorem}
\newtheorem{defin}[theorem]{Definition}
\newtheorem{nota}[theorem]{Notation}
\newtheorem{fact}[theorem]{Fact}
\newtheorem{case}[theorem]{Case}



\crefname{thm}{Theorem}{Theorems}
\Crefname{thm}{Theorem}{Theorems}
\crefname{problem}{Problem}{Theorems}
\Crefname{problem}{Problem}{Theorems}
\Crefname{assump}{Assumption}{Theorems}
\crefname{assump}{Assumption}{Theorems}
\crefname{conjecture}{Conjecture}{Theorems}
\Crefname{conjecture}{Conjecture}{Theorems}
\crefname{prop}{Proposition}{Propositions}
\Crefname{prop}{Proposition}{Propositions}
\crefname{cor}{Corollary}{Corollaries}
\Crefname{cor}{Corollary}{Corollaries}
\crefname{lem}{Lemma}{Lemmas}
\Crefname{lem}{Lemma}{Lemmas}
\theoremstyle{definition}
\crefname{defn}{definition}{definitions}
\Crefname{defn}{Definition}{Definitions}
\crefname{conj}{Conjecture}{Conjectures}
\Crefname{conj}{Conjecture}{Conjectures}
\crefname{remark}{Remark}{Remarks}
\Crefname{remark}{Remark}{Remarks}
\crefname{rmk}{Remark}{Remarks}
\Crefname{rmk}{Remark}{Remarks}
\crefname{example}{Example}{Examples}
\Crefname{example}{Example}{Examples}
\Crefname{case}{Case}{Cases}
\Crefname{Case}{Case}{Cases}
\crefname{align}{}{}
\Crefname{align}{}{}
\crefname{equation}{}{}
\Crefname{equation}{}{}





%\newcounter{count}

%\newcommand{\cbr}[1]{\left\{ #1 \right\}}
%\newcommand{\rbr}[1]{\left( #1 \right)}
%\newcommand{\sbr}[1]{\left[ #1 \right]}
\newcommand{\textdef}[1]{\textit{#1}\index{#1}}
\newcommand{\sS}{\mathcal{S}}
%\newcommand{\QQP}{(QQP)}
%\newcommand{\SDP}{(SDP)}

%\newcommand{\Xo}{{X_{\rm opt}}}
%\newcommand{\Xoa}{X_{{\rm opt},\alpha}}
%\newcommand{\yo}{{y_{\rm opt}}}
%\newcommand{\Zo}{{Z_{\rm opt}}}
\newcommand{\Xa}{{X(\alpha)}}
\newcommand{\ya}{{y(\alpha)}}
\newcommand{\Za}{{Z(\alpha)}}
\newcommand{\ao}{{\alpha_0}}
\newcommand{\po}{{p_{\rm opt}}}
\newcommand{\Xpa}{{X^{\prime}(\alpha)}}
\newcommand{\ypa}{{y^{\prime}(\alpha)}}
\newcommand{\Zpa}{{Z^{\prime}(\alpha)}}


\newcommand{\KK}{{\mathcal K} }
\newcommand{\cK}{{\mathcal K} }
\newcommand{\cS}{{\mathcal S} }
\newcommand{\ZZ}{{\mathcal Z} }
\newcommand{\XX}{{\mathcal X} }
\newcommand{\ZS}{{\mathcal Z_S} }
\newcommand{\Xs}{{\mathcal X}_s }
\newcommand{\Lx}{{L}_X }
\newcommand{\XS}{{\mathcal W} }
%\newcommand{\Ss}{{\mathcal S} }
\newcommand{\Ekyb}{{\mathcal E}^{n}(1{:}k,\bar D)}
\newcommand{\Eayb}{{\mathcal E}^{n}(\alpha,\bar D)}
\newcommand{\Ls}{{L}_S }
\newcommand{\E}{{\mathcal E} }
\newcommand{\LL}{{\mathcal L} }
\newcommand{\RR}{{\mathcal R} }
\newcommand{\YY}{{\mathcal Y} }
\newcommand{\cY}{{\mathcal Y} }
\newcommand{\EE}{{\mathcal E} }
\newcommand{\FF}{{\mathcal F} }
\newcommand{\DD}{{\mathcal D} }
\newcommand{\HH}{{\mathcal H} }
\newcommand{\BB}{{\mathcal B} }
\newcommand{\cB}{{\mathcal B} }
\newcommand{\CC}{{\mathcal C} }
\newcommand{\GG}{{\mathcal G} }
\newcommand{\OO}{{\mathcal O} }
\newcommand{\PP}{{\mathcal P} }
\newcommand{\PZA}{{\PP_{\ZZ_A}}}
\newcommand{\Nn}{{\NN}^n}
\newcommand{\TT}{{\mathcal T} }
\newcommand{\UU}{{\mathcal U} }
\newcommand{\VV}{{\mathcal V} }
\newcommand{\cV}{{\mathcal V} }
\newcommand{\WW}{{\mathcal W} }
\newcommand{\uh}{\hat u}
\newcommand{\vh}{\hat v}
\newcommand{\Uh}{\hat U}
\newcommand{\SC}{\mathcal{S}^n_C}
\newcommand{\Sh}{{\mathcal S}^n_H}
\newcommand{\Scal}{{\mathcal S}}
\newcommand{\snn}{{\mathcal S}_{n-1} }
\newcommand{\Snn}{{\mathcal S}^{2n+1} }
\newcommand{\Snm}{{\mathcal S}^{n-1} }
\newcommand{\Snmp}{{\mathcal S}^{n-1}_+ }
\newcommand{\Sd}{{\mathcal S}^{d} }
\newcommand{\Sdp}{{\mathcal S}^{d}_{+} }
\newcommand{\Sdpp}{{\mathcal S}^{d}_{++} }
\newcommand{\Snnp}{{\mathcal S}^{n+1}_+ }
\newcommand{\Snnpp}{{\mathcal S}^{n+1}_{++} }
\newcommand{\CP}{{\bf{\rm CP\,}}}
\newcommand{\CPD}{{\bf{\rm CPD\,}}}
\newcommand{\LCP}{{\bf{\rm LCP\,}}}
\newcommand{\LCPE}{{\bf{\rm LCPE\,}}}
\newcommand{\LCDE}{{\bf{\rm LCDE\,}}}
\newcommand{\hs}{H_{\sigma}}
\newcommand{\En}{{{\mathcal E}^n} }
\newcommand{\Ek}{{{\mathcal E}^k} }
\newcommand{\Kdag}{{\KK}^{\dagger}}
\newcommand{\KKi}{{\KK}^i}
\newcommand{\KdagV}{{\KK_V}^{\dagger}}
\newcommand{\KV}{{\KK}_V}

\newcommand{\whV}{{\widehat{V}} }

\newcommand{\cL}{{\mathcal L} }
\newcommand{\cG}{{\mathcal G} }
\newcommand{\cI}{{\mathcal I} }
\newcommand{\cJ}{{\mathcal J} }

\newcommand{\Sayb}{{\mathcal S}^{n}(\alpha,\bar Y)}
\newcommand{\Sapyb}{{\mathcal S}_+^{n}(\alpha,\bar Y)}
\newcommand{\Skyb}{{\mathcal S}^{n}(1{:}k,\bar Y)}
\newcommand{\Skpyb}{{\mathcal S}_+^{n}(1{:}k,\bar Y)}
\newcommand{\Sok}{{\mathcal S}^{1{:}k}}

\newcommand{\QED}{{\flushright{\hfill ~\rule[-1pt] {8pt}{8pt}\par\medskip ~~}}}
\newcommand{\mat}{{\rm mat}}


\def\Rf{\R^f}
\def\Rfp{\R_+^f}
\def\Rnf{\R^{nf}}
\def\Rnfp{\R_+^{nf}}
\newcommand{\NTnf}{NT_\text{nf}}
\newcommand{\NTf}{NT_\text{f}}
\newcommand{\ATnf}{AT_\text{{nf}}}
\newcommand{\ATf}{AT_\text{f}}
\newcommand{\xf}{x_\text{f}}
\newcommand{\xs}{x_\text{s}}
\newcommand{\xmineq}{x_{\mineq}}
\newcommand{\xhf}{\hat x_\text{f}}
\newcommand{\xnf}{x_\text{nf}}
\newcommand{\xhnf}{\hat x_\text{nf}}
\newcommand{\Xnf}{X_\text{nf}}
\newcommand{\zf}{z_\text{f}}
\newcommand{\cf}{c_\text{f}}
\newcommand{\Af}{A_\text{f}}
\newcommand{\Anf}{A_\text{nf}}
\newcommand{\Bf}{B_\text{f}}
\newcommand{\Bnf}{B_\text{nf}}
\newcommand{\Ef}{E_\text{f}}
\newcommand{\Enf}{E_\text{nf}}
\newcommand{\znf}{z_\text{nf}}
\newcommand{\Znf}{Z_\text{nf}}
\newcommand{\vng}{v_\text{ng}}
\newcommand{\vg}{v_\text{g}}
\newcommand{\cnf}{c_\text{nf}}
\newcommand{\bineq}{b_{\text{ineq}}}
\newcommand{\beq}{b_{\text{eq}}}
\newcommand{\bub}{b_{\text{ub}}}
\newcommand{\Aeq}{A_{\text{eq}}}
\newcommand{\Aineq}{A_{\text{ineq}}}
\newcommand{\mineq}{m_\text{ineq}}
\newcommand{\meq}{m_\text{eq}}
\newcommand{\Fmudnf}{F_\mu^{\text{dnf}}}
\newcommand{\Fmudf}{F_\mu^{\text{df}}}
\newcommand{\Fmupnf}{F_\mu^{\text{pnf}}}
\newcommand{\Fmupf}{F_\mu^{\text{pf}}}
\newcommand{\Fmucnf}{F_\mu^{\text{cnf}}}




%\newcommand{\ip}[2]{\left\langle #1, #2 \right\rangle}
\newcommand{\set}[1]{\left\{ #1 \right\}}
\newcommand{\cbr}[1]{\left\{ #1 \right\}}
\newcommand{\rbr}[1]{\left( #1 \right)}
\newcommand{\sbr}[1]{\left[ #1 \right]}
%\newcommand{\norm}[1]{\left\| #1 \right\|}
\newcommand{\rPRSM}{\textbf{rPRSM}\,} % changed by TK
\newcommand{\rPRSMp}{\textbf{rPRSM}}
\newcommand{\QP}{\textbf{QP}\,}
\newcommand{\QPp}{\textbf{QP}}
\newcommand{\QAP}{\textbf{QAP}\,}
\newcommand{\QAPp}{\textbf{QAP}}
\newcommand{\HKM}{\textbf{HKM}\,}
\newcommand{\HKMp}{\textbf{HKM}}
\newcommand{\DR}{\textbf{DR}\,}
\newcommand{\DRp}{\textbf{DR}}
\newcommand{\ADMM}{\textbf{ADMM}\,}
\newcommand{\ADMMp}{\textbf{ADMM}}
\newcommand{\PR}{\textbf{PR}\,}
\newcommand{\PRp}{\textbf{PR}}
\newcommand{\PRS}{\textbf{PRS}\,}
\newcommand{\PRSM}{\textbf{PRSM}\,}
\newcommand{\PRSMp}{\textbf{PRSM}}
\newcommand{\DNN}{\textbf{DNN}\,}
\newcommand{\DNNp}{\textbf{DNN}}
\newcommand{\KKT}{\textbf{KKT}\,}
\newcommand{\KKTp}{\textbf{KKT}}
\newcommand{\BFS}{\textbf{BFS}\,}
\newcommand{\BFSp}{\textbf{BFS}}
\newcommand{\EDM}{\textbf{EDM}\,}
\newcommand{\EDMp}{\textbf{EDM}}
\newcommand{\SDP}{\textbf{SDP}\,}
\newcommand{\SDPs}{\textbf{SDPs}\,}
\newcommand{\SDPsp}{\textbf{SDPs}}
\newcommand{\SDPR}{\textbf{SDP-R}\,}
\newcommand{\NDS}{\textbf{NDS}\,}
\newcommand{\NDSp}{\textbf{NDS}}
\newcommand{\DS}{\textbf{DS}\,}
\newcommand{\DSp}{\textbf{DS}}
\newcommand{\DSDP}{\textbf{D-SDP}\,}
\newcommand{\DSDPp}{\textbf{D-SDP}}
\newcommand{\SDPp}{\textbf{SDP}}
\newcommand{\DFR}{\textbf{DFR}\,}
\newcommand{\DFRp}{\textbf{DFR}}
\newcommand{\FR}{\textbf{FR}\,}
\newcommand{\FRp}{\textbf{FR}}
\newcommand{\FDC}{{\bf{\rm FDC\,}}}
\newcommand{\SOC}{{\bf{\rm SOC\,}}}
\newcommand{\UCQ}{{\bf{\rm UCQ\,}}}
\newcommand{\WCQ}{{\bf{\rm WCQ\,}}}
\newcommand{\UCQs}{{\bf{\rm UCQs\,}}}
\newcommand{\WUCQ}{{\bf{\rm WUCQ\,}}}
\newcommand{\CQ}{{\bf{\rm CQ\,}}}
\newcommand{\CQc}{{\bf{\rm CQ},\,}}
\newcommand{\CQs}{{\bf{\rm CQs\,}}}
\newcommand{\SDPns}{{\bf{\rm SDP}}}
\newcommand{\DLP}{{\bf{\rm DLP\,}}}
\newcommand{\DLPp}{{\bf{\rm DLP}}}
\newcommand{\LP}{{\bf{\rm LP\,}}}
\newcommand{\LPp}{{\bf{\rm LP}}}
\newcommand{\LPns}{{\bf{\rm LP}}}
\newcommand{\Ksym}{{\stackrel{s}{\otimes}}}
\newcommand{\Kt}{{\stackrel{T}{\otimes}}}
\newcommand{\Ksksym}{{\stackrel{sk}{\otimes}}}
\newcommand{\Ksumsym}{{\stackrel{s}{\oplus}}}
\newcommand{\symmkron}{{{O\hspace{-.09in}\cdot}\hspace{.05in}}}
\newcommand{\cond}{{\rm cond\,}}
\newcommand{\gap}{{\rm gap\,}}
\newcommand{\optval}{{\rm optval\,}}
\newcommand{\Rm}{{\R^m\,}}
%\newcommand{\Sn}{{\mathcal S^n\,}}
%\newcommand{\Sn}{{\mathcal S}^n}
\newcommand{\SN}{{\mathcal S^N}}
\newcommand{\SNp}{{\mathcal S^N_+}}
\newcommand{\Ht}{{\mathbb H}^t}
\newcommand{\Hm}{{\mathbb H}^m}
\newcommand{\Hmn}{{\mathbb H}^{mn}}
\newcommand{\Hmnp}{{\mathbb H}_+^{mn}}
%\newcommand{\Snp}{{\mathcal S^n_+\,}}
%\newcommand{\Snp}{{\mathcal S}^n_+\,}
%\newcommand{\Stt}{\Sc^2_+}
%\newcommand{\Snpp}{{\mathcal S^n_{++}\,}}
%\newcommand{\Srpp}{{\mathcal S^r_{++}\,}}
\newcommand{\Rtn}{{\R^{\scriptsize{t(n)}}\,}}
\newcommand{\Ctn}{{\C^{\scriptsize{t(n)}}\,}}
%\newcommand{\Stn}{{{\mathcal S}^{\scriptsize{\begin{pmatrix}n\cr 2}}}\,\end{pmatrix}}
\newcommand{\Sk}{{\mathcal S^{k}}\,}
\newcommand{\Skp}{{\mathcal S^{k}_+}\,}
\newcommand{\Skpp}{{\mathcal S^{k}_{++}}\,}
\newcommand{\Stp}{{\mathcal S^{t}_+}\,}
\newcommand{\Stpp}{{\mathcal S^{t}_{++}}\,}
\newcommand{\Stnn}{{{\mathcal S}^{\scriptsize{t(n)}}\,}}
\newcommand{\Mtnn}{{{\mathcal M}^{\scriptsize{t(n)}}\,}}
\newcommand{\tn}{{\scriptsize{\pmatrix{n\cr 2}}}\,}
\newcommand{\tnp}{{\scriptsize{\pmatrix{n+1\cr 2}}}\,}
\newcommand{\Rtnp}{{\R^{\scriptsize{\pmatrix{n+1\cr 2}}}\,}}
\newcommand{\MM}{{\mathcal M}}
\newcommand{\Mn}{{\mathcal M}^n}
\newcommand{\MS}{{\mathcal M_S\,}}
\newcommand{\NS}{{\mathcal N_S\,}}
\newcommand{\NN}{{\mathcal{N}}}
\newcommand{\cN}{{\mathcal{N}}}
\newcommand{\MSi}{{\mathcal M_S^{-1}\,}}
\newcommand{\Mrid}{M_{\real\imag D}}
\newcommand{\Mridbar}{\overline M_{\real\imag D}}
\newcommand{\Cp}{{\mathcal C\,}}
\newcommand{\p}{{\mathcal P\,}}
\newcommand{\A}{{\mathcal A}}
\newcommand{\B}{{\mathcal B\,}}
\newcommand{\II}{{\mathcal I}}
\newcommand{\JJ}{{\mathcal J}}
\newcommand{\IIf}{{\mathcal I^f}}
\newcommand{\IIp}{{\mathcal I^+}}
\newcommand{\IIn}{{\mathcal I^-}}
\newcommand{\IIo}{{\mathcal I^0}}
\newcommand{\XSA}{{{\mathcal W}_{\A}} }
\newcommand{\F}{{\mathcal F\,}}
\newcommand{\U}{{\mathcal U\,}}
\newcommand{\Uone}{{\mathcal U_1\,}}
\newcommand{\QQ}{{\mathcal Q\,}}
\newcommand{\Flift}{{\mathcal F}_n}
% \renewcommand{\theequation}{\thesection.\thecount}
\newcommand{\bbm}{\begin{bmatrix}}
\newcommand{\ebm}{\end{bmatrix}}
\newcommand{\bem}{\begin{pmatrix}}
\newcommand{\eem}{\end{pmatrix}}
%\newcommand{\beq}{\begin{equation}} % conflict here 
\newcommand{\beqs}{\begin{equation*}}
\newcommand{\bet}{\begin{table}}
%\newcommand{\bet}{\begin{table}}
\newcommand{\eeq}{\end{equation}}
\newcommand{\eeqs}{\end{equation*}}
%\newcommand{\beqr}{\begin{eqnarray}}
\newcommand{\beqr}{\begin{eqnarray}}
% \newcommand{\addc}{\addtocounter{count}{1} }
% newcommand{\bs}{\setcounter{count}{0} \section}
%\newcommand{\bs}{\section}
\newcommand{\fa}{~ \; \forall}
\newcommand{\req}[1]{(\ref{#1})}
\newcommand{\sn}{{\mathcal S}_n }
\newcommand{\hn}{{\mathcal H}_n }
\newcommand{\g}{{\mathcal G} }
\newcommand{\ind}{{\mathcal Ind} }
\newcommand{\SKn}{{\tilde{\mathcal S}}^n}
\newcommand{\Rtwon}{\mathbb{R}^{2n}}

\renewcommand{\vec}{{\rm vec}}

\DeclareMathOperator{\prox}{{\text{prox}} }
\DeclareMathOperator{\refl}{{\text{refl}} }


\DeclareMathOperator{\face}{face}% smallest face containing a set
\DeclareMathOperator{\facee}{face^{ef}}% smallest face containing a set
\DeclareMathOperator{\sd}{sd}

%\newcommand{\epreq}{\hfill $\blacksquare$}
\DeclareMathOperator{\Null}{null}
\DeclareMathOperator{\nul}{null}
\DeclareMathOperator{\range}{range}
\DeclareMathOperator{\embdim}{embdim}
\DeclareMathOperator{\dist}{dist}
\DeclareMathOperator{\kvec}{{vec}}
\DeclareMathOperator{\kMat}{{Mat}}
\DeclareMathOperator{\adj}{{adj}}
\DeclareMathOperator{\vol}{{vol}}
\DeclareMathOperator{\btrace}{{blocktrace}}
\DeclareMathOperator{\blkdiag}{{blkdiag}}
\DeclareMathOperator{\Blkdiag}{{Blkdiag}}
\DeclareMathOperator{\diag}{{diag}}
\DeclareMathOperator{\Diag}{{Diag}}
\DeclareMathOperator{\dDiag}{{dDiag}}
\DeclareMathOperator{\Ddiag}{{Ddiag}}
\DeclareMathOperator{\Mat}{{Mat}}
\DeclareMathOperator{\offDiag}{{offDiag}}
\DeclareMathOperator{\usMat}{{us2Mat}}
\DeclareMathOperator{\usvec}{{us2vec}}
\DeclareMathOperator{\svec}{{svec}}
\DeclareMathOperator{\ssvec}{{ssvec}}
\DeclareMathOperator{\sHvec}{{sHvec}}
\DeclareMathOperator{\sbHvec}{{\bar sHvec}}
\DeclareMathOperator{\sblkHvec}{{sblkHvec}}
\DeclareMathOperator{\sblkHMat}{{sblkHMat}}
\DeclareMathOperator{\vsvec}{{vs2vec}}
\DeclareMathOperator{\dsvec}{{dsvec}}
\DeclareMathOperator{\Hmat}{{Hmat}}
\DeclareMathOperator{\vSmat}{{vSmat}}
\DeclareMathOperator{\Smat}{{S2mat}}
\DeclareMathOperator{\sMat}{{sMat}}
\DeclareMathOperator{\sHMat}{{sHMat}}
\DeclareMathOperator{\hMat}{{hMat}}
\DeclareMathOperator{\vsMat}{{vs2Mat}}
\DeclareMathOperator{\kmat}{{Mat}}
\DeclareMathOperator{\sdiag}{{sdiag}}
\DeclareMathOperator{\Kprod}{\otimes}
\DeclareMathOperator{\Kcal}{{\cal K}}
\DeclareMathOperator{\Hcal}{{\cal H}}
\DeclareMathOperator{\relint}{{relint}}
\DeclareMathOperator{\bdry}{{bdry}}
\DeclareMathOperator{\cl}{{cl}}
%\DeclareMathOperator{\abs}{{abs}}
%\DeclareMathOperator{\rank}{{rank}}
\DeclareMathOperator{\spanl}{{span}}
\DeclareMathOperator{\dom}{{dom}}
\DeclareMathOperator{\conv}{{conv}}
%\DeclareMathOperator{\real}{{\Re}}
\DeclareMathOperator{\imag}{{\Im}}
\DeclareMathOperator{\sign}{{sgn}}
%\DeclareMathOperator{\odiag}{{$o\circ$ diag}}
%\DeclareMathOperator{\bdiag}{{b$^\circ$diag}}




\newcommand{\Mod}[1]{\ (\textup{mod}\ #1)}



%   for Real number sign
%\newcommand{\eqref}[1]{{\rm (\ref{#1})}}
\newcommand{\gesem}{\succeq}
\newcommand{\lesem}{\preceq}
\newcommand{\Sbar}{\bar{S}}
\newcommand{\Qbar}{\bar{Q}}
\newcommand{\Tbar}{\bar{T}}
\newcommand{\xbar}{\bar{x}}
\newcommand{\Xbar}{\bar{X}}
\newcommand{\nc}{\newcommand}
\nc{\arrow}{{\rm arrow\,}}
\nc{\Arrow}{{\rm Arrow\,}}
\nc{\BoDiag}{{\rm B^0Diag\,}}
\nc{\bodiag}{{\rm b^0diag\,}}
\newcommand{\oodiag}{{\rm o^0diag\,}}
\newcommand{\OoDiag}{{\rm O^0Diag\,}}
\def\nr{\par \noindent}
\nc{\Mm}{{\mathcal M}^{m} }
\nc{\Mmn}{{\mathcal M}^{mn} }
\nc{\Mnr}{{\mathcal M}_{nr} }
\nc{\Mnmr}{{\mathcal M}_{(n-1)r} }
% \nc{\eqref}[1]{{\rm (\ref{#1})}}
\nc{\kwqqp}{Q{$^2$}P\,}
\nc{\kwqqps}{Q{$^2$}Ps}
%\def\Argmin{\mathop{\rm Argmin}}
%\def\Argmax{\mathop{\rm Argmax}}


\def\Ssum{\mathop{\mathcal S_\Sigma}}
\def\Sprod{\mathop{\mathcal S_\Pi}}
\nc{\notinaho}{(X,S)\in \overline{AHO}(\A)}
\nc{\inaho}{(X,S)\in AHO(\A)}

% Simon's definitions
\newcommand{\bea}{\begin{eqnarray}}%
\newcommand{\eea}{\end{eqnarray}}%
\newcommand{\beas}{\begin{eqnarray*}}%
\newcommand{\eeas}{\end{eqnarray*}}%
\newcommand{\Rmn}{\R^{m \times n}}%
\newcommand{\Rnn}{\R^{n \times n}}%
\newcommand{\Rnnp}{\R_+^{n \times n}}%
\newcommand{\Int}{{\rm int\,}}% interior of a set
\newcommand{\cone}{{\rm cone\, }}% cone generated by a set
\newcommand{\sspan}{{\rm span}\,}
\newcommand{\ccone}{{\rm \overline{cone}}}% cone generated by a set
\newcommand{\precl}{{\rm precl}}% preclosure of a set
%\newcommand{\bdry}{{\rm bdry}}% boundary of a set
%\newcommand{\cl}{{\rm cl}}% closure of a set
\newcommand{\bnd}{{\rm bnd}}% boundary of a set
\newcommand{\relbnd}{{\rm relbnd\,}}% relative boundary of a set
%\newcommand{\refl}{{\rm refl}}%
\newcommand{\re}{{\rm Re}\,}%
\newcommand{\im}{{\rm Im}\,}%
%\newcommand{\Range}{\RR}%
\newcommand{\la}{\ensuremath{\Leftarrow}}%
\newcommand{\ra}{\ensuremath{\Rightarrow}}%
\newcommand{\lla}{\;\ensuremath{\Longleftarrow}\;}% 
\newcommand{\rra}{\;\ensuremath{\Longrightarrow}\;}% 
\newcommand{\lra}{\ensuremath{\Leftrightarrow}}% 
\newcommand{\llra}{\; \Longleftrightarrow\;}% 
\renewcommand{\F}{\mathcal{F}}%
\renewcommand{\L}{\mathcal{L}}%
\newcommand{\X}{\mathcal{X}}%
\newcommand{\nn}{\nonumber}%
\newcommand{\vP}{v_P}%
\newcommand{\vD}{v_D}%
\newcommand{\vDP}{v_D}%
\newcommand{\vRP}{v_{RP}}%
\newcommand{\vDRP}{v_{DRP}}%
\newcommand{\vDRPM}{v_{DRP_M}}%
% Produce array delimited by []. Usage: \arr{ccc}{B&-A^T&C\\I&0&A}
\newcommand{\arr}[2]{\ensuremath{
\left[\begin{array}{#1}#2\end{array}\right]%
}}{}
% Produce small array (within text)
% e.g.,~$\bigl[\begin{smallmatrix} A&C\\0&-I \end{smallmatrix}\bigr]$
\newcommand{\sarr}[1]{\ensuremath{
\left[\begin{smallmatrix}#1\end{smallmatrix}\right]}}
% Produce *nondelimited* small array underneath a min or max 
% e.g.,~$\begin{smallmatrix} x \in X\\ y \in Y \end{smallmatrix}$
\newcommand{\tarr}[1]{\ensuremath{
\begin{smallmatrix} #1 \end{smallmatrix}}}
% Relabel list environments.
% undo this relabel - by Henry????? - comment out the next four lines????
%\makeatletter
%\renewcommand{\theenumi}{\rm (\alph{enumi})}
%\renewcommand{\p@enumii}{\theenumi}
%\makeatother

%%from Vris
\newcommand{\Hnp}[1][]{\,\mathbb{H}_+^{\ifthenelse{\equal{#1}{}}{n}{#1}}}
\newcommand{\Hn}[1][]{\,\mathbb{H}^{\ifthenelse{\equal{#1}{}}{n}{#1}}}
\newcommand{\Hk}[1][]{\,\mathbb{H}^{\ifthenelse{\equal{#1}{}}{k}{#1}}}
\newcommand{\Dn}[1][]{\,\mathbb{D}^{\ifthenelse{\equal{#1}{}}{n}{#1}}}
\newcommand{\inprod}[1]{\langle#1\rangle}
\newcommand{\seq}[2]{#1^{(#2)}} % \seq{x}{n} gives x^{(n)}
\newcommand{\Tcal}{\mathcal{T}}
\newcommand{\TP}{\mathrm{TP}}
\newcommand{\unital}{\mathrm{unital}}

\newcommand{\bmat}[1]{\begin{bmatrix} #1\end{bmatrix}}
%\newcommand{\pmat}[1]{\begin{pmatrix} #1\end{pmatrix}}
\newcommand{\ii}{\mathrm{i}\,}

%for HugoHenryStefan FacialReduction
\newcommand{\FFp}{{\mathcal F}^+ }
\newcommand{\ba}{b(\alpha) }
\newcommand{\Pa}{{\bf P(\alpha)}}
\newcommand{\pa}{{\mathcal P(\alpha)}}
\newcommand{\Ppa}{{\bf P}_{\alpha}'}
\newcommand{\PVa}{{\bf P}_{\alpha}^{V}}
\newcommand{\Da}{{\bf D_{\alpha}}}
\newcommand{\Fa}{\F(\alpha)}
\newcommand{\Fap}{\FF_{\alpha}^+}
\newcommand{\eig}{\text{eig}}

\newcommand{\aff}{\text{aff}}
\newcommand{\xaff}{x_\text{aff}}
\newcommand{\yaff}{y_\text{aff}}
\newcommand{\zaff}{z_\text{aff}}
\newcommand{\Xaff}{X_\text{aff}}
\newcommand{\Zaff}{Z_\text{aff}}


% Naomi's definitions
\newcommand{\cP}{\mathcal{P}}

\newcommand{\<}{\langle}
\renewcommand{\>}{\rangle}

%%%%%%%%%%%%%%%%%%%%%%%%%%%%%%%%%%%%%%%%%%%%%%%%%%%%%%%%%%%%%%%%%%%%%%%%%%%%%%%%%%%%%%%%%
%%%%%%%%%%%%%%%%%%%%%%%%%%%%%%%%%%%%%%%%%%%%%%%%%%%%%%%%%%%%%%%%%%%%%%%%%%%%%%%%%%%%%%%%%

\begin{document}

{\bf CO 367, F23 \hfill  ASSIGNMENT~0 (optional)\footnote{
\tiny COPYRIGHT: Reproduction, sharing or online posting of this document is strictly
 forbidden $\copyright$ University of Waterloo.
}
 \hfill Due Thurs. Sept. 14, 9:30AM, EST on crowdmark}
\medskip

\hrule

Please submit a pdf file to CROWDMARK. No other form of submission is 
acceptable; late assignments will not be accepted. It is the 
responsibility of the students to make sure that the pdf file they 
submit is clearly readable.

\noindent
Important information: 
\\{\bf You are expected to justify your answers, e.g.,~a yes or no, a
number, a property, etc ... is not enough.} 
\\{\bf You are expected to do the assignments on 
your own. The only sources
allowed for doing the assignments are:
\begin{itemize}
\item
the lectures presented in class;
\item
all of the material on the CO367 Fall 2023 LEARN website;
\item
the two textbooks as listed on the syllabus (and on library reserve);
\item
 all of the material on the CO367 Fall 2023 Piazza website;
\item
 your Instructor and TAs.
\end{itemize}

Usage of any other sources in doing the homework assignments is against the
academic integrity rules for CO367 in the Fall 2023 term.
}

For example, copying or paraphrasing a solution from some fellow student or from old solutions from
previous offerings of related courses or various sources on the web, qualifies as cheating and the TAs have
been instructed to actively look for evidence of academic offences when evaluating assignment papers. By
Univ. of Waterloo policy, academic integrity violations by a student in assignments will lead to a mark of zero in that
assignment and potentially a 5\% deduction from the Final Course Grade. All academic offences are reported
to the Associate Dean for Undergraduate Studies.

[This assignment is a \emph{Calculus, Analysis, Topology, 
Linear Algebra Review} 
for topics
most relevant to CO367. If you have trouble completing
any part of this assignment it is a strong indication that you need to
review the relevant material.]


\newpage



The University of Waterloo subscribes to the strictest interpretation of
academic integrity.

{\bf Faculty members and students bear joint responsibility in assuring
 that cheating on assignments or any examinations is not tolerated.
}

Students who engage in academic dishonesty will be subject to disciplinary action under Policy 71 Student
Discipline, 
\href{https://uwaterloo.ca/secretariat/policies-procedures-guidelines/policy-71}
{uwaterloo.ca/secretariat/policies-procedures-guidelines/policy-71}


\vspace{2in}
I confirm that I have not violated the instructions for this Assignment and I will not violate
the instructions for this Assignment.
I confirm that I have not received any unauthorized assistance in preparing for or writing
this Assignment.


Student Name (please print)
\\Student Signature

\newpage

\tableofcontents



\newpage
\section{GEOMETRY
{\small(score 20 = 10+10)}
}


\subsection{Preliminaries}
For a vector $x=\begin{pmatrix} x_1 & \ldots & x_n\end{pmatrix}^T \in \Rn$, 
we define the (Euclidean) {\em norm} and {\em inner (dot) product}
\[  
\|x\| := \sqrt{x_1^2+ \cdots+ x_n^2}
\]
\[  \langle x,y\rangle \cong x \cdot y := x^Ty = x_1y_1+ \cdots+ x_ny_n.
\]
\begin{theorem}
\label{thm:CauchySchwarz}
 Let $x,y\in \Rn$. Then the Cauchy-Schwarz inequality is
\[
|x \cdot y| \leq ||x||||y||,
\]
with equality if, and only if, $x=\alpha y$, for some $\alpha\in \R$.
\end{theorem}
\subsection{EXERCISES}
\begin{enumerate}
\item (10)
Explain the geometrical significance for the vectors $x$ and $y$ if:
\begin{enumerate}
\item
\[x \cdot y < 0, \, x \cdot y > 0, \, x \cdot y = 0.
\]
(Justify/prove the explanations.)
%%\begin{solution}
%%The vectors are orthogonal (perpendicular). For a proof use the cosine
%%formula for the angle $\theta$ between the vectors
%%\[
%%\cos(\theta) = (x\cdot y)/\left( \|x\|\|y\|\right).
%%\]
%%\end{solution}
%%\begin{comments}
%%Many students did not include a proof. Justification must always be included.
%%\end{comments}
\item
\[x \cdot y \geq 0. \]
%\begin{solution}
%The angle between the vectors is acute, i.e.,~less than 90 degrees. Again
%this can be seen from the cosine formula.
%\end{solution}
%\begin{comments}
%Many students did not include a proof. Justification must always be included.
%\end{comments}
\end{enumerate}
\item
(10) 
Let $a\in \Rn, Q\in \Sn$ and define the (quadratic)
function $f: \Rn \rightarrow \R$ by 
$f(x) = \frac 12 x^TQx + a \cdot x$.
Prove that $f$ is a continuous differentiable function, i.e.,~that the
gradient $\nabla f(x)$ is a continuous function of $x$.
(See~\cref{eq:gradient} below.)
%%\begin{solution}
%%Let $\epsilon >0$ be given, and fix $\delta = \epsilon/(2\|a\|)$. Then
%%by~\Cref{thm:CauchySchwarz}, we  have
%%\[
%%\|x-y\| <\delta \implies |f(x) - f(y)| = |a\cdot(x-y)|\leq \|a\| \|x-y\| <
%%\epsilon.
%%\]
%%\qed
%%\end{solution}
%\begin{comments}
 %Some students thought this was a one dimensional case and used
%elementary techniques on the real line.
%\end{comments}
\end{enumerate}




\section{CALCULUS
{\small(score 30 = (5+10+5)+(5+5))}
}
\subsection{EXERCISES}
\begin{enumerate}
\item 
For a differentiable function $g: \Rn \rightarrow \R$, 
the {\em gradient} is the vector of partial derivatives
\begin{equation}
\label{eq:gradient}
\nabla g := \begin{pmatrix}
   \frac{\partial g}{\partial x_1}\\ 
   \frac{\partial g}{\partial x_2}\\ 
     \cdots\\ \frac{\partial
g}{\partial x_n}
\end{pmatrix}.
\end{equation}
\begin{enumerate}
\item 
\label{item:fng}
(5)
If $g(x) = || x ||,\, x\neq 0$, calculate $\nabla g(x)$.
%\begin{solution}
%$g(x) = || x || = \sqrt{x^Tx} = f(h(x))$, thus defining $f,h$. Therefore
%\[
%\nabla g(x) = f^\prime(h(x)) \nabla h(x) = \frac 1{2\sqrt{h(x)}} 2x =
 %\frac 1{\norm{x}} x.
%\]
%\end{solution}
\item (10)
Let $g$ be defined as in~\Cref{item:fng} above.
Let $a,b \in \Rn, (a+tb)\neq 0$, and define $f:\R \rightarrow \R,$ by
$f(t) := g(a+tb)$.
\\Calculate $f^\prime (t)$.
%%\begin{solution}
%%(Here $f$ is not as in the proof above.)
%%\[
%%f^\prime(t) = \left\langle \nabla g(a+tb), \frac {\partial (a+tb)}{\partial
%%t}\right\rangle = \frac 1{\norm{a+tb}} \langle a+tb,b\rangle.
%%\]
%%\end{solution}
%\begin{comments}
%Students often did not use the chain rule correctly.
%\end{comments}
\item
(5) Let $g$ be as above and let $x,d\in Rn$. Calculate the directional
derivative of $g$ in the direction $d$ at the point $x$.
\end{enumerate}
\item
Suppose $f:\R \rightarrow \R$ is infinitely differentiable at $x=a.$
The {\em Taylor Series} of $f$ about $a$ is:
\[
f(a) + f^\prime(a) (x-a) + \frac1{2!}f^{\prime \prime}(a)(x-a)^2 + 
    \frac1{3!}f^{\prime \prime \prime}(a)(x-a)^3 + \cdots
\]
Write down the Taylor series of:
\begin{enumerate}
\item (5)
\[
f(x) = x^3, \mbox{~about~} x=1.
\]
\item (5)
\[
f(x) = \log(1+x), \mbox{~about~} x=0.
\]
\end{enumerate}
\end{enumerate}
\section{REAL ANALYSIS
{\small(score 35 = 15+(10+10))}
}
\subsection{Preliminaries}
The {\em open ball} $B(x;r) : = \left\{ y \in \Rn : ||x-y|| < r
\right\}.$ Suppose that $D$ is a subset of $\Rn$.
\begin{description}
\item[Interior:]
$x \in \Int D$ if there exists $r > 0$ with $B(x;r) \subset D$.
\item[Closure:]
$x \in \cl D$ if there exists a sequence $x^k \in D$ with $x^k
\rightarrow x.$
\item[Boundary:]
$x \in \partial D$ if $x\in \cl D \backslash \Int D.$
\end{description}
$D$ is {\em open} if $D = \Int D.$ 
$D$ is {\em closed} if $D = \cl D.$ \\

\subsection{EXERCISES}
\begin{enumerate}
\item
(15)  For each of the following sets, find the interior, the
closure, and the boundary. Then determine which of the sets are open,
closed, neither, or both.
%\begin{solution}
%For more details please contact the instructor or the TAs.
%\end{solution}
\begin{enumerate}
\item
\[
\left\{
(x_1,x_2) : x_1 \geq 0, x_2 \geq 0
\right\}.
\]
%\begin{solution}
%closed
%\end{solution}
\item
\[
\left\{
(x_1,x_2) : x_1 > 0, x_2 > 0
\right\}.
\]
%\begin{solution}
%open
%\end{solution}
\item
\[
\left\{
(x_1,x_2) : x_1 > 0, x_2 \geq 0
\right\}.
\]
%\begin{solution}
%neither
%\end{solution}
\item  
\[
\Rn
\]
%\begin{solution}
%both
%\end{solution}
\item
\[
\left\{
(x_1,x_2) : x_1^2 + x_2^2 < 0
\right\}.
\]
%\begin{solution}
%both (empty set)
%\end{solution}
\item
\[
\emptyset .
\]
%\begin{solution}
%both (empty set)
%\end{solution}
\end{enumerate}
%\begin{comments}
 %For computing the boundary in part a)b)c), many students
%forgot that the sets were part of the nonnegative quadrant.
%\end{comments}
\item
\begin{enumerate}
\item (10) Prove that $D$ is closed if and only if the complement $D^c$
is open.
%\begin{solution}
%Suppose that the complement $D^c$ is open. Therefore, for each $c\in
%D^c$ there exists an open ball $B(c,r_c)\subset D^c$. Now let $d_j\in D,
%d_j\to d$. Therefore, for each ball $B(d,r)$ around $d$ there exists $J$ such
%that 
%\[
%\left\{ j\geq J\implies d_j \in B(d,r)\right\}  \implies
%d\notin D^c \text{ (open set) } \implies d\in D \implies D \text{ is
%closed}.
%\] 
%(For the converse, see the TAs or the instructor.)
%\end{solution}
\item (10)
Prove that $x \in \partial D$ if and only if for any $r>0$ there exists
a $y \in B(x;r) \cap D$ and a $z \in B(x;r) \cap D^c.$
%\begin{solution}
%(See the TAs or the instructor.)
%\end{solution}
\end{enumerate}
%\begin{comments}
%Many proofs lacked rigour.
%\end{comments}

\end{enumerate}



\section{MATRICES 
{\small(score 25 = 15+10)}
}

\subsection{EXERCISES}
\begin{enumerate}
\item (15)
Let
\[ 
A = 
\begin{bmatrix}
1 & 1 & 0 \\
0 & 1 & 1 \\
1 & 2 & 1 
 \end{bmatrix}.
\]

\begin{enumerate}
\item
Calculate the determinant of $A$.
\item
Calculate the rank of $A$.
\item
What is the rank of $A^T$.
\item
Find the singular values and singular vectors of $A$.
\item
Expand the quadratic form $q(x) = x^TAx = \sum_{ij} x_ix_j B_{ij}$ so
that $B=(B_{ij}) = B^T$, i.e.,~so that the form is using a symmetric
matrix.
\end{enumerate}
  
%%\begin{solution}
%%(See the TAs or the instructor.)
%%\begin{enumerate}
%%\item
%%The determinant is $0$.
%%\item
%%$2$, see below.
%%\item
%%$\rank A = \rank A^T$
%%\item
%%the singular values are $3,1,0$ and so the rank is $2$.
%%The singular vectors can be found using the (square root of) 
%%eigenvalues, eigenvectors
%%of $A^TA, AA^T$, since $AA^T = U\Sigma V^T V\Sigma U^T = U\Sigma^2U^T$ and
%%similarly for $A^TA$.
%%\begin{comments}
%%The square root was often forgotten for the eigenvalues. Also the
%%left/right eigenvectors were not used appropriately.
%%\end{comments}
%%\end{enumerate}
%%\end{solution}

\item (10)
Let
\[ 
C = 
 \begin{bmatrix}
-2 & 1 & 0\cr
0 & -1 & 2\cr
30 & -18 & 9
 \end{bmatrix}.
\]
Calculate the eigenvalues and eigenvectors of $C$. Show your
calculations.
\\Hint: Let $v = \begin{pmatrix}1& 3 & 3\end{pmatrix}^T$ and evaluate
$Cv$.

\end{enumerate}
%%\begin{solution}
%%(See the TAs or the instructor.)
%%Note that the trace is the sum of the eigenvalues while the determinant
%%is the product. Therefore solve the two equations for the eigenvalues
%%\[
%%\lambda_1+\lambda_2 = 4;\, \lambda_1\lambda_2 = 3.
%%\]
%%Once the eigenvalues $(1,3)$ are found,
%%the eigenvectors can be found from solving the linear systems
%%$Bv_i = \lambda_iv_i,\, i=1,2$ to get $(1,-1)^T,\,(-1,-1)^T$,
%%respectively.
%%\end{solution}

\end{document}



