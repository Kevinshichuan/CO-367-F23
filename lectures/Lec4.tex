\subsection{Lecture 4}

\begin{definition}[Principal Submatrices]
  Let $$A = \begin{bmatrix}
    1 & 1 & 2 & 7 \\
    1 & 1 & 4 & 6 \\
    2 & 4 & 7 & 8 \\
    7 & 6 & 8 & 1 \\
  \end{bmatrix}, \quad I = \{1,3\}, \quad A[I] = \begin{bmatrix}
    1 & 2 \\
    2 & 7 \\
  \end{bmatrix}$$
  Then $A[I]$ is a \textbf{principal submatrix} of $A$.
\end{definition}
\begin{definition}[Principal Minors]
  Let $A \in \mathbb S^n$, where $\mathbb S^n$ is the set of all symmetric $n \times n$ matrices.
\begin{enumerate}
  \item $\det(A[I])$ is called the \textbf{principal minor} of $A$.
  \item If $I = \{1,\ldots, k\}$ then $\det(A[I])$ is called the \textbf{leading principal minor} of $A$.
\end{enumerate}
\end{definition}
\begin{proposition}[Characterizing Positive Definiteness with Principal Minors]
  Let $A \in \mathbb S^n$. Then
  \begin{enumerate}
    \item $A \succeq 0 \iff \det(A[I]) \geq 0$ for all principal minors $\det(A[I])$.
    \item $A \succ 0 \iff \det(A[I]) > 0$ for all \textbf{leading} principal minors $\det(A[I])$.
  \end{enumerate}
\end{proposition}
\begin{definition}[Eigenvectors and Eigenvalues]
  $0 \neq v \in \mathbb R^n$ is an \textbf{eigenvector} of $A$ if there exists $\lambda \in \mathbb R$ such that $Av = \lambda v$. The number $\lambda$ is called an \textbf{eigenvalue} of $A$.
\end{definition}
\begin{theorem}[Finding Eigenvectors and Eigenvalues]
  Let $A$ be a matrix.
  \begin{enumerate}
    \item Set up the characteristic equation. We find $$\det(A - \lambda I) = 0$$
    \item Solve for $\lambda$. These are the eigenvalues.
    \item Plug eigenvalues $\lambda_1,\ldots, \lambda_n$ into $(A - \lambda I)v = 0$ and solve for $v$. These are the eigenvectors.
  \end{enumerate}
\end{theorem}
\begin{theorem}[Orthogonal Spectral Decomposition]
  Let $A \in \mathbb S^n$. Then $A$ has an \textbf{orthogonal spectral decomposition}
  $$A = \sum_i \lambda_i u_i u_i^T = UDU^T$$ where $U$ is orthogonal with the orthogonal eigenvectors $u_i$ as columns and $D$ is a diagonal matrix with real eigenvalues on the diagonal.
\end{theorem}
\begin{corollary}
  Let $A \in \mathbb S^n$. Then
  \begin{enumerate}
    \item $A \succeq 0$ (positive semidefinite) iff all eigenvalues of $A$ are nonnegative.
    \item $A \succ 0$ (positive definite) iff all eigenvalues of $A$ are positive.
  \end{enumerate}
\end{corollary}
\begin{proposition}
  Let $A \in \mathbb S^n$. The following are equivalent (Positive definite):
  \begin{enumerate}
    \item $A \succ 0$.
    \item All the eigenvalues of $A$ are in $\mathbb R^n_{++}$, the interior of the nonnegative orthant.
    \item $A$ has a real symmetric positive definite square root, $A = SS, S \in \mathbb S^n_{++}$.
    \item $A$ has a lower triangular factorization, a Cholesky factorization, $A = LL^T$ and $L$ has positive diagonal elements.
    \item All principal minors of $A$ are positive.
    \item All leading principal minors of $A$ are positive.
  \end{enumerate}
  And the following are equivalent (Positive semidefinite):
  \begin{enumerate}
    \item $A \succeq 0$.
    \item All the eigenvalues of $A$ are in $\mathbb R^n_{+}$, the nonnegative orthant.
    \item $A$ has a real symmetric square root, $A = SS, S \in \mathbb S^n$.
    \item $A$ has a lower triangular factorization, a Cholesky factorization, $A = LL^T$.
    \item All principal minors of $A$ are nonnegative.
  \end{enumerate}
\end{proposition}
\begin{problem}[Motivation]
  When can we guarantee that global minimizers of $f: \mathbb R^n \rightarrow \mathbb R$ exist?

  For example, the real valued function on $\mathbb R$ $f(x) = e^x$ is bounded below by 0 but has no minimizers. The minimum value is 0 but is not attained.
\end{problem}
\begin{proposition}[Weierstrass Extreme Value Theorem]
  If $f: \mathbb R^n \rightarrow \mathbb R$ is continuous, and if $D \subset \mathbb R^n$ is a closed and bounded set, then $f$ is bounded below and the minimum value is attained on $D$.
\end{proposition}
\begin{definition}[Coercive function]
  A continuous function $f: \mathbb R^n \rightarrow \mathbb R$ is \textbf{coercive} if for any sequence $x_i$ with $\|x_i \| \rightarrow \infty$, it must be the case that $f(x_i) \rightarrow +\infty$. In other words, $$\lim_{\|x\| \rightarrow \infty} f(x) = +\infty$$
  Here are some examples:
  \begin{enumerate}
    \item $f_1(x) = x^2$ is coercive.
    \item $g(x) = x$ is not coercive (because as $x \rightarrow -\infty$, $g(x) \rightarrow -\infty \neq \infty$).
    \item $h(x) = e^x$ is not coercive.
  \end{enumerate}
\end{definition}
\begin{proposition}[Coercive Functions and Minimizers]
  A coercive function $f: \mathbb R^n \rightarrow \mathbb R$ has a global minimizer.
\end{proposition}
\begin{definition}[Level Sets]
  Let $f: \mathbb R^n \to \mathbb R$ be a function and let $\alpha \in \mathbb R$. An $\alpha$-level set of $f$ is defined by $$L_\alpha = \{x \in \mathbb R^n: f(x) = \alpha\}$$
  That is, all points $x$ such that $f(x) = \alpha$.
  \begin{itemize}
    \item When $n = 2$, we call this a level curve.
    \item When $n = 3$, we call this a level surface.
    \item When $n > 3$, we call this a level hypersurface.
  \end{itemize}
\end{definition}
\begin{definition}[Sub-level set]
  Let $f: \mathbb R^n \to \mathbb R$ be a function and let $\alpha \in \mathbb R$. An $\alpha$-sublevel set of $f$ is defined by $$S_\alpha(f) = \{x \in \mathbb R^n: f(x) \leq \alpha\}$$
  That is, all points $x$ below the line $f(x) = \alpha$.
\end{definition}

\begin{problem}
  Can we find minimizer for $q:\mathbb{R^n}\rightarrow \mathbb{R}$.
  $q(x)=\frac{1}{2}x^TQx+b^Tx+\alpha(Q=Q^T)$
  
  Let $p^*=inf q(x)$ 
  \begin{itemize}
    \item When is it finite?
    \item When is it attained?
  \end{itemize} 
  
\end{problem}

\begin{proposition}[bdd blow \textcolor{red}{Needtoconfirm}]
   $q(x)$ is bdd below iff $Q\succeq 0$ and b$\in$ Range(Q) and 0=$\nabla q(x)=Qx-b$ so $x^*=Q^{-1}b$ is Orthogonal (\textcolor{red}{why}) so always attained


\end{proposition}
\begin{proof}
  Since Q is positive semidef, we can use spectural decomposition,
  $Q=UDU^T,U^TU=I$. $x^TQx=x^TUDU^Tx$ which is also a quadratic form, let $y=U^Tx,x=Uy$
  \\ Thus for q(x), we sub $x=uy$, then we get 
  \begin{align*}
    q(x) &= \frac{1}{2}(Uy)^TQ(Uy)+b^TUy+\alpha \\
    &= \frac{1}{2}y^TDy+(U^Tb)^Ty+\alpha \text{  Let }\bar{b}=U^Tb \\
    &= \sum\lambda_iy_i^2+\bar{b_i}y_i+\alpha
  \end{align*}
  Which is a seperable problem.
  $p_i^*=min\, \frac{1}{2}\lambda_iy_i^2+\bar{b_i}y_i$
  \\ $p_i^*$ is finite iff $\lambda_i \geq 0,\implies . $\textcolor{red}{This part incomplete}
\end{proof}

\begin{theorem}[Weierstrass]
  Given f:$R^n\rightarrow R$ cts and D in $R^n$ closed and bounded,
  Then $\exists\bar{x}\in D$s.t $\bar{x}\in$ argmin f(x)
\end{theorem}

\begin{proposition}
  If f is coercive, f maps $R^n$ to R and cts, then there exists $\bar{x}\in argmin\,f(x)$
\end{proposition}
\begin{proof}
  Omit
\end{proof}

\begin{theorem}[q(x) is coercive iff Q is positive definite]
  
\end{theorem}
\begin{proof}
  Omit
\end{proof}