
\subsection{Lecture 15}
\subsubsection{Convex Functions}
\begin{definition}[Epigraph]
    Let $f: \mathbb R^n \to \mathbb R$. Then the epigraph of $f$ is defined as
    $$\text{epi}(f) = \{(x, r): f(x) \leq r\}$$
    $f$ is convex if and only if $\text{epi}(f)$ is a convex set.
\end{definition}

\begin{definition}[Convex Functions and Strictly Convex Functions/Zero Order Characterization Convex Function]
    Let $f: C \to \mathbb R$ where $C \subseteq \mathbb R^n$ is a convex set. We say that $f$ is a convex function if for every $x, y \in C$ and for every $\lambda \in [0,1]$,
    $$f((1 - \lambda)x + \lambda y) \leq (1 - \lambda) f(x) + \lambda f(y)$$
    We say that $f$ is strictly convex if for every $x, y \in C, x \neq y$ and for every $\lambda \in [0,1]$,
    $$f((1 - \lambda)x + \lambda y) < (1 - \lambda) f(x) + \lambda f(y)$$
    If the above is an equality, then $f$ is an affine function.
\end{definition}
\begin{definition}[Concave Functions]
    The function $f$ is called concave if $-f$ is convex.
\end{definition}
\begin{theorem}[Affine Function]
    Suppose that $f: \mathbb R^n \to \mathbb R$ is both concave and convex. Then $f$ is affine.
\end{theorem}
\begin{definition}[Unit Simplex]
    The unit simplex is defined as 
    $$\Delta_k = \{\lambda \in \mathbb R^n_+: e^T\lambda = 1\}$$
    where $e = (1, \ldots, 1)^T \in \mathbb R^n$.
\end{definition}
\begin{theorem}[Jensen's Inequality]
    Let $C \subseteq \mathbb R^n$ be a convex set and $f: C \to \mathbb R$ be a convex function. Then for any $x_i \in C, i = 1,\ldots,k$ and $\lambda \in \Delta_k$, we have
    $$f\left(\sum_i^k \lambda_{i=1} x_i\right) \leq \sum_{i=1}^k \lambda_i f(x_i)$$
\end{theorem}
\begin{theorem}[]
    Let $C$ be an open convex set in $\mathbb R^n$ and $f: C \to \mathbb R$ is a convex function. Then $f$ is continuous. 
\end{theorem}
\begin{theorem}[First Order Condition]
    Assume that $f(x)$ is continuously differentiable on a convex set $C \subseteq \mathbb R^n$. Then the function $f(x)$ is
    \begin{enumerate}
        \item convex if and only if for all $x, y \in C$, $$f(x) + \nabla f(x)^T (y - x) \leq f(y)$$
        \item strictly convex if and only if for all $x, y \in C$ with $x \neq y$,
        $$f(x) + \nabla f(x)^T (y - x) < f(y)$$
    \end{enumerate}
\end{theorem}
\begin{proof}[Proof]
    See past course notes
\end{proof}
\begin{theorem}[Second Order Condition]
    Assume that $f(x)$ is twice continuously differentiable on an open convex set $C \subseteq \mathbb R^n$. Then the function $f(x)$ is convex if and only if $\nabla^2 f(x)$ is positive semidefinite for every $x \in C$.
\end{theorem}
\begin{proof}[Proof]
    See past course notes
\end{proof}
\begin{corollary}[Critical Points of Convex Functions]
    Let $f$ be convex on convex set $C$, $\bar x \in C$. Then $$\nabla f(\bar x) = 0 \iff f(\bar x) \leq f(y)$$ for all $y \in C$ (So $\bar x$ is a global minimizer of $f$ on $C$).

    So, $\bar x$ is a critical point of $f$ if and only if it is a global minimizer of $f$ on $C$.
\end{corollary}\
\begin{lemma}
    Let $C \subseteq \mathbb R^n$, $C$ convex. Let $f: C \to \mathbb R$ be a $C^1$-smooth convex function on $C$. Then for $x,y \in C$, $$f(y) < f(x) \implies \underbrace{f'(x; y - x)}_{\nabla f(x)^T (y - x)} < 0$$
\end{lemma}
\begin{theorem}[]
    Let $C \subseteq \mathbb R^n$ be a convex set. Then any local minimizer (in $C$) of a convex function $f: C \to \mathbb R$ is also a global minimizer of $f$ on $C$. Furthermore, any local minimizer of a strictly convex function $f: C \to \mathbb R$ is the unique global minimizer of $f$ on $C$.
\end{theorem}
\begin{lemma}[Global Maximizer of Convex Functions]
    Let $f: C \to \mathbb R$, $f$ convex, $C\subseteq \mathbb R^n$ convex. Let $\bar x, \bar y \in C$ with $$\bar x \in \text{argmax}_{x \in C}f(x), \quad f(\bar x) > f(\bar y)$$ then $\bar x \not \in \text{int}(C)$ (Must be on boundary).

    So a global maximum of a convex function on a convex set must be on the boundary of the set.
\end{lemma}
\subsubsection{Operation that Preserve Convexity}
Course notes