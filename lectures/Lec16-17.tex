\newpage
\section{Constrained Optimization}
\subsection{Lectures 16-17}
\subsubsection{Linear Programming, Lagrange Multipliers, and Implicit Function Theorem}
\begin{example}[Simplex Method]
    Consider the following LP in standard form:
    \begin{align*}
        \min \quad & c^T x \\
        \text{s.t.} \quad & Ax = b \in \mathbb R^n\\
        & x \geq 0
    \end{align*}
    where $A$ has full row rank (constraints are in linear independence (LICQ)).

    Let $x = \begin{pmatrix}x_B \\ x_N \end{pmatrix}$ denote a basic feasible solution, BFS. So $x$ is feasible and $x_N = 0$. 
    
    We call $x_B$ the basic variables and $x_N$ the non-basic variables. 
    
    We let $A = \begin{pmatrix}B & N\end{pmatrix}$ where $B$, the basis matrix, is an $m \times m$ invertible matrix. W.L.O.G. we have assumed the first $m$ variables to be the basic variables, so the basis is $B = \{1,\ldots,m\}$.

    We split the constraints $Ax = b$ into two parts, $$Bx_B + Nx_N = b$$ 
    We solve our equation w.r.t. $x_B$ to get $$x_B = B^{-1}b - B^{-1}Nx_N$$
    Similarly, we can split our objective function $c^T x$ into two parts, $$c^T_B x_B + c^T_N x_N$$ Substituting $x_B = B^{-1}b - B^{-1}Nx_N$ into objective function gives
    $$c^T_B x_B + c^T_N x_N = c^T_B(B^{-1}b - B^{-1}Nx_N) + c^T_N x_N$$
    Then let $y^T = c^T_BB^{-1}$ denote the dual variable estimates. The objective function becomes
    \begin{align*}
        c^Tx &= y^Tb - y^TNx_N + c^T_Nx_N \\
        &= y^Tb + (c^T_N - y^TN)x_N
    \end{align*}
    We can see that the objective values decreases if we find a reduced cost $c^T_N - y^TN < 0$ and increase the corresponding nonbasic variable $x_j$, i.e. $j$ enters the basis. Review Simplex method if needed.
\end{example}
\begin{example}
    Example of Lagrange multipliers and implicit function thoerem here that I do not understand.
\end{example}
\begin{theorem}[Lagrange Multipliers for Equality Constraints]
    Suppose that $f: C \to \mathbb R$ and $h: C \to \mathbb R^m$ are sufficiently smooth functions on the open set $C \subseteq \mathbb R^n$. Let $x \in C$ be a local minimum of $f$ subject to the constraints $h(x) = 0, x \in C$. In addition, assume the following regularity condition at $x$ (constraint qualification at $x$) $$h'(x) \text{ is onto}$$
    Then there exists $\lambda \in \mathbb R^m$ (a Lagrange multiplier vector) such that $$0 = \nabla f(x) + \langle \lambda, h'(x) \rangle$$ Equivalently, the gradient $\nabla f(x)$ is in the range of $h'(x)^T$.
\end{theorem}
\subsection{Optimality Conditions and Geometry}
Consider the following LP:
\begin{align*}
    \min \quad & f(x) \\
    \text{s.t.} \quad & x \in Q
\end{align*}
where $f: \mathbb R^n \to \mathbb R$ is $C^1$ or $C^2$ smooth and $Q$ is a closed set.
\begin{example}
    If we have the LP
    \begin{align*}
        \min \quad & f(x) \\
        \text{s.t.} \quad & c_1(x_1,x_2) = x_1^2 + x_2^2 = 0 
    \end{align*}
    Let $x^*$ be a local optimal solution. Then $\nabla f(x^*) \in \mathrm{span}(\nabla c_i)$ by Lagrange Multiplier since
    $$\nabla f(x^*) + \lambda \nabla c_1(x^*) = 0$$ and rearranging gives
    $$\nabla f(x^*) = -\lambda \nabla c_1(x^*)$$
\end{example} 
\begin{example}
    Consider the LP
    \begin{align*}
        \min \quad & f(x) = x_1 + x_2 \\
        \text{s.t.} \quad & c_1(x_1,x_2) = 1 - x_1^2 - x_2^2 \geq 0 
    \end{align*}
    Let $\bar x$ be a local optimal solution, then
    \begin{enumerate}
        \item If $\bar x$ satisfies $c_1(\bar x) > 1$, then $\nabla f(\bar x) = 0$.
        \item If $\bar x$ satisfies $c_1(\bar x) = 1$, then $\nabla f(\bar x) \in \underbrace{\mathrm{cone}\{\nabla c_1(\bar x)\}}_{-\lambda \nabla c_1(\bar x), \lambda \leq 0}$.
    \end{enumerate}
\end{example}
\begin{definition}[Tangent Vector]
    Let $\bar x \in Q \subseteq \mathbb R^n$. A vector $v \in \mathbb R^n$ is tangent to $Q$ at $\bar x \in Q$ if there exists sequences $\{x_k\}\subseteq Q$ and $\{t_k\} \subset R_+$ such that $$x_k \to \bar x, \quad \underbrace{t_k \downarrow 0}_{t_k > 0, t_k \to 0, \text{non-increasing}}, \quad \frac{1}{t_k}(x_k - \bar x) \to v$$
\end{definition}
\begin{definition}[Tangent Cone]
    We denote the set of all tangent vectors to $Q$ at $\bar x$ as the tangent cone $T_Q(\bar x)$.
\end{definition}