\subsection{Lecture 11+ cts}
To finish:

Newton/trust region (second order model)

\begin{definition}[p-th order convergence]
    If residuals $r_R\to 0$ and 
    $lim\frac{||r_{k+1}||}{||r_k||^p}\to r >0$ is called p-th order convergence.
    \\ If p=1,r<1,then this is linear convergence, (For s.d, linear , r $\cong r$ (cond(Hessian at $x^*$)))
    \\ If p =2, quadratic convergence
\end{definition}

\begin{theorem}[Convergence of Newton]
    Suppose $f:R^n\to R\,in\, c^2-Smooth$, $\nabla^2f(x^*)>0,\nabla f(x^2)=0$, and 
    the hessian locally lipschitz continuous.\\
    Consider iterately:
    $$x_{k+1}=x_k+t_kv_k,v_k=-\nabla^2f(x_k)^{-1}\nabla f(x_k)(Newton\,direction)$$
    Where $t_k$ satisfy wolfe condition, then  if the ""\textcolor{red}{text} point $x_0$,
    "close enough" to $x^*$ ...Theorem 5.6.2 \textcolor{red}{text finish it after}
\end{theorem}
\begin{problem}
    $d_{SC}=-\nabla f(x_k),d_N=-\nabla^2f(x_k)^{-1}\nabla f(x_k) $
\end{problem}

%\begin{Problem}
    levenberg marquardt algorithm
$-(\nabla^2 f(x_k)+\lambda I)^{-1}\nabla f(x_k)$ $\lambda >0$
%\end{Problem}
\begin{problem}
    This leads to trust region method ,
    follows by 5.8.1 Trust Region Methods Outline.
    note: Find $d_k$ is trust region subproblem
\end{problem}


\section{Convex set and function}
\subsection{Lecture 11+ cts}
\begin{definition}[Line Segment]
    Let $x,y \in \mathbb R^n$. The line segment joining $x$ and $y$ is the set
    $$[x,y] = \{\lambda x + (1 - \lambda)y : \lambda \in [0,1]\}$$
\end{definition}

\begin{definition}[Convex set]
    Set $S$ in $R^n$ in a convex set if for every $x, y \in S$ and for every  $0\leq \lambda \leq 1$ $$\lambda x+(1-\lambda)y\in S$$
    That is, the line segment $[x,y]$ is contained in $S$.
\end{definition}
\begin{definition}[Convex combination]
    Let $S$ be a convex set and let $x_1,\ldots x_n \in S$. If $\lambda_1,\lambda_2,\ldots,\lambda_n$ are nonnegative numbers such that $$\sum_{i=1}^{n} \lambda_i = 1$$ then $$\sum^n_{i=1} \lambda_i x_i$$ is called a convex combination and is contained in $S$.
\end{definition}

\begin{lemma}
    $S$ is a convex set iff $S$ contains all its convex combination
\end{lemma}
\begin{definition}
    Given $S\in R^n$, the convex hull of $S$, $\text{conv }S$ is the smallest convex set containing $S$ and is equal to the set of all convex combinations of points in $S$.
\end{definition}

\begin{lemma}
    Suppose $S$ is a compact set, then $\text{conv }S$ is a compact set.
\end{lemma}