% \documentclass[../main.tex]{subfiles}

% \begin{document}
\section{Introduction}
\subsection{Lecture 1-Preliminaries}
\begin{definition}[Quadratic Form]
    Let $A$ be a symmetric matrix and $x = \begin{pmatrix}
        x_1 \\ \vdots \\ x_n
    \end{pmatrix}$. The \textbf{quadratic form} $Q$ of the matrix $A$ is defined as $$Q = x^T Ax$$
\end{definition}

\begin{problem}[Example]
    Consider the matrix $A = \begin{bmatrix}
        5 & -5 \\
        -5 & 1
    \end{bmatrix}$. The quadratic form of $A$ is $$Q(x) = 5x^2_1 - 10x_1x_2 + x^2_2$$
\end{problem}
\begin{definition}[Classification of Quadratic Forms]
    Let $Q$ be a quadratic form of a matrix $A$. Then $Q$ is
    \begin{enumerate}
        \item positive definite if $Q(x) > 0$ for all non-zero vectors $x$, and $Q(x) = 0$ if and only if $x = 0$. Or all eigenvalues of $A$ are positive.
        \item positive semidefinite if $Q(x) \geq 0$ for all vectors $x$, with $Q(x) = 0$ occurring for some non-zero vectors $x$. Or all eigenvalues of $A$ are non-negative.
        \item negative definite if $Q(x) < 0$ for all non-zero vectors $x$, and $Q(x) = 0$ if and only if $x = 0$. Or all eigenvalues of $A$ are negative.
        \item negative semidefinite if $Q(x) \leq 0$ for all vectors $x$, with $Q(x) = 0$ occuring for some non-zero vectors $x$. Or all eigenvalues of $A$ are non-negative.
        \item indefinite if $Q(x)$ can be positive or negative. Or there are positive and negative eigenvalues for $A$.
    \end{enumerate}
\end{definition}

\begin{definition}[\href{https://www.stat.cmu.edu/~cshalizi/uADA/13/lectures/app-b.pdf}{Big O and little o}]
    Big O is basically the rate of growth of that function. A function $f(n)$ is of order 1, or $O(1)$ if there exists some non zero constant $c$ such that $$\frac{f(n)}{c} \rightarrow 1$$ as $n \rightarrow \infty$.

    Little o is the upper bound of the rate of growth of that function. Therefore, a function $f(n)$ is of order 1, or $o(1)$ if for all constants $c > 0$, $$\frac{f(n)}{c} \rightarrow 0$$ as $n \rightarrow \infty$.
\end{definition}

\begin{definition}[\href{https://sites.math.washington.edu/~folland/Math134/lin-approx.pdf}{Differentiability Based on Big o and Little o}]
    If $f$ is differentiable at $x = a$, then $$f(a + h) = f(a) + f'(a)h + o(h)$$
    Conversely, if there exists constants $A$ and $B$ such that $$f(a + h) = A + Bh + o(h)$$ then $f$ is differentiable at $x = a$. Moreover, $A = f(a)$ and $B = f'(a)$.
\end{definition}
\begin{definition}[Product Rule]
    If $f, g$ are differentiable at $x = a$, then $$f(a + h) = f(a) + f'(a)h + o(h), \quad g(a + h) = g(a) + g'(a)h + o(h)$$
    Then
    \begin{align*}
        p(a + h) &= f(a + h)g(a + h) \\
        &= f(a)g(a) + [f(a)g'(a) + g(a)f(a)]h + o(h) 
    \end{align*}
    Then by above theorem, $p = fg$ is differentiable at $x = a$, and $p'(a) = f(a)g'(a) + g(a)f'(a)$.
\end{definition}
\begin{definition}[Chain Rule]
    WIP
\end{definition}



\begin{definition}[Inner Product Space]
    Let $x \in \mathbb{R}^n$, represented as:
    
    \[
    x = \begin{pmatrix}
        x_1 \\
        \vdots \\
        x_n
    \end{pmatrix}
    \]
    
    The inner product space is defined as:
    
    \[
    \langle x, y \rangle = x^T y = \sum_{i=1}^{n} x_i y_i \quad \text{(dot product)}
    \]

    The angle between vectors $x$ and $y$ is given by $\cos(\theta) = \frac{\langle x, y \rangle}{\|x\|}$.

    With corresponding norm to be the Euclidean Norm
\end{definition}

\begin{definition}[Open ball]
    Given $\delta > 0$, $\bar{x}\in \mathbb{R}^n$, the open ball $B_{\delta}(\bar{x})=\{x \in \mathbb{R}^n\ | \|x-\bar{x}\|< \delta\}$
    
\end{definition}

\begin{definition}[map]
    Suppose the map $f:\mathbb{R}^n -> \mathbb{R}$.
\end{definition}

\begin{definition}[open set]
    Let D $\subset \mathbb{R}^n$, D open set. $\forall x \in D$, $\exists \delta >0$, s.t $B_{\delta}(x)\subset D$
\end{definition}

\begin{definition}[differ]
    We define \(f\) to be in \(C^1, C^2\) on an open set \(D \subseteq \mathbb{R}^n\), denoted \(f \in C^1(D), C^2(D)\), respectively, if the partial first \(\frac{\partial f(x)}{\partial x_i}\) and second \(\frac{\partial^2 f(x)}{\partial x_i \partial x_j}\) derivatives exist and are continuous for all \(i, j\), respectively. We then get the gradient vector in \(\mathbb{R}^n\) and the \(n \times n\) symmetric Hessian matrix, respectively denoted as:
\[
\nabla f(x) = \left( \frac{\partial f(x)}{\partial x_i} \right) \in \mathbb{R}^n, \quad \nabla^2 f(x) = \left[ \frac{\partial^2 f(x)}{\partial x_i \partial x_j} \right] \in \mathbb{S}^n.
\]
Here, \(\mathbb{S}^n\) is the vector space of \(n \times n\) symmetric matrices.

\end{definition}


\begin{definition}[General Nonlinear opt. function NLO]
    The general problem of nonlinear optimization, denoted NLO, is defined as follows: Given C²-smooth functions
\(f, g_i, h_j : D \subseteq \mathbb{R}^n \rightarrow \mathbb{R}\) for \(i = 1, \ldots, m\) and \(j = 1, \ldots, p\), where \(D\) is an open subset of \(\mathbb{R}^n\), the objective is to find the optimal value \(p^*\) and an optimum \(x^*\) of NLO, represented as:

\[
p^* := \min f(x) \\
\text{s.t.} \quad g_i(x) \leq 0, \quad \forall i = 1, \ldots, m \\
h_j(x) = 0, \quad \forall j = 1, \ldots, p \\
x \in D
\]

If $f,g_i,h_i$ are all \textbf{affine} function and D=$R^2$, then we have an LP
\end{definition}

\begin{definition}[affine]
    % Define the affine function: f(x) = Ax + b
\begin{equation}
    f(x) = Ax + b
\end{equation}
where b$\neq$0

\end{definition}
\begin{definition}[Types of Minimality]
    Consider \(f : \mathbb{R}^n \rightarrow \mathbb{R}\) and \(D \subset \mathbb{R}^n\). Then \(\bar{x} \in D\) is:

\begin{itemize}
    \item a \textit{global minimizer} for \(f\) on \(D\) if \(f(\bar{x}) \leq f(x)\) for all \(x \in D\).
    \item a \textit{strict global minimizer} for \(f\) on \(D\) if \(f(\bar{x}) < f(x)\) for all \(x \in D\) where \(x \neq \bar{x}\).
    \item a \textit{local minimizer} for \(f\) on \(D\) if there exists \(\delta > 0\) such that \(f(\bar{x}) \leq f(x)\) for all \(x \in D \cap B_\delta(\bar{x})\).
    \item a \textit{strict local minimizer} for \(f\) on \(D\) if there exists \(\delta > 0\) such that \(f(\bar{x}) < f(x)\) for all \(x \in D \cap B_\delta(\bar{x})\) where \(x \neq \bar{x}\).
\end{itemize}
\end{definition}
\begin{definition}[Linear Approximation]
    Suppose $f$ is a function that is differentiable on an interval $I$ containing the point $a$. The \textbf{linear approximation} to $f$ at $a$ is the linear function $$L(x) = f(a) + f'(a)(x-a)$$ for $x \in I$.
\end{definition}
\begin{definition}[Quadratic Approximation]
    Similar as above, the \textbf{quadratic approximation} to $f$ at $a$ is the quadratic function $$Q(x) = f(a) + f'(a)(x-a) + \frac{1}{2}f''(a)(x-a)^2$$ for $x \in I$.
\end{definition}
\begin{definition}[Formal Definition of Derivative]
    The \textbf{derivative} of $f$ at $a$ is defined as $$f'(a) = \lim_{x \rightarrow a} \frac{f(x) - f(a)}{x-a}$$ if the limit exists.

    An alternate definition is
    $$f'(a) = \lim_{h \rightarrow 0} \frac{f(a+h) - f(a)}{h}$$
\end{definition}
% \end{document}

