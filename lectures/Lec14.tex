\subsection{Lecture 14}
\subsubsection{Review of Linear Programming}
Review of basic solutions, basic feasible solutions, simplex etc. Basic feasible solutions are extreme points.
\subsubsection{Operations that Preserve Convexity}
\begin{itemize}
    \item If $Q_i$ are convex, then $\bigcap_i Q_i$ is convex (can be infinite).
    \item For an affine map $f: C \to D$, $f(x) = Ax + b$, where $C, D$ are convex. The images $f(C)$, and preimages $f^{-1}(D)$ are convex sets.
    \item Projections of convex sets $P(C)$ are convex.
    \item Scaling of convex sets are convex, $\alpha C$
    \item Rotation of convex sets are convex, $QC, Q^TQ = I$.
    \item Translation of convex sets are convex, $C + x$.
\end{itemize}
\subsubsection{Hyperplane Separation and Support Theorems}
\begin{theorem}[Hyperplane Separation Theorem]
    Let $C, D \subset \mathbb R^n$ be convex sets and $C \cap D = \emptyset$. Then there exists $a \neq 0, \alpha \in \mathbb R$ such that $$a^Tx \leq \alpha \leq a^Ty$$ for all $x \in C, y \in D$.

    \bigskip
    In other words, this theorem guarantees that if you have two sets that do not intersect and are convex, there is a hyperplane (which is a generalization of a flat surface to $n$-dimensions; for instance, a line in 2D or a plane in 3D) that separates the two sets. This hyperplane can be thought of as a decision boundary, which one can use to distinguish points belonging to one set from those belonging to the other.
\end{theorem}
\begin{proof}[Proof]
    See studocu course notes
\end{proof}
\begin{definition}[Minkowski Sum and Difference]
    Let $C, D \subseteq \mathbb R^n$ be two sets. The Minkowski sum of $C$ and $D$ is defined as
    $$C + D = \{x + y: x \in C, y \in D\}$$
    The Minkowski difference of $C$ and $D$ is defined as
    $$C - D = \{x - y: x \in C, y \in D\}$$
\end{definition}
\begin{lemma}
    Let $C \subset \mathbb R^n$ be a closed convex set and let $d \not \in C$. Let $x^*$ be the nearest point in $C$ to $d$. Then the hyperplane $$H = \{x: \langle v, x \rangle = \langle v, d \rangle - \|v\|^2\}$$ strictly separates $C$ from $d$. That is, we have
    $$\langle v, x \rangle \leq \langle v, d \rangle - \|v\|^2 < \langle v, d \rangle$$ for all $x \in C$.
\end{lemma}
\begin{definition}[Supporting Hyperplane]
    $H = \{x: v^Tx = b\}$ is a supporting hyperplane to a convex set $Q$ at $\bar x \in Q$, if 
    $$x^Tv \leq b , \quad \forall x \in Q$$ and the equation $x^T v = b$ holds.
\end{definition}



