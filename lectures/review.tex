\section{Review}
\subsection{Calculus}
\begin{definition}[Norm]
  $\| \cdot \|: \mathbb R^n \to \mathbb R$ is a mapping from $\mathbb R^n$ to $\mathbb R$ such that
  \begin{itemize}
    \item For every $\x \in \mathbb R^n$, $\| \x \| \geq 0$
    \item $\| \x \| = 0$ if and only if $\x = 0$.
    \item For every $\alpha \in \mathbb R$ and for every $\x \in \mathbb R^n$, $\| \alpha \x \| = |\alpha| \| \x \|$
    \item (Triangle inequality) For every $\textbf{x, y} \in \mathbb R^n$, $$\| \x + \y \| \leq \| \x \| + \| \y \|$$
  \end{itemize}
  Two common norms, $\ell_1$ and $\ell_2$ norms:
  \begin{enumerate}
    \item $\ell_1$: $\| \x \|_1 = \sum_{i=1}^n |x_i| = |x_1| + |x_2| + \cdots + |x_n|$
    \item $\ell_2$: $\| \x \|_2 = \sqrt{\sum_{i=1}^n x_i^2} = \sqrt{x_1^2 + x_2^2 + \cdots + x_n^2}$
  \end{enumerate}
\end{definition}
\begin{proposition}[Cauchy-Schwarz Inequality]
  For every $\x,\y \in \mathbb R^n$
  $$|\x^T \y| = \langle \x, \y \rangle \leq \| \x \|_2 \| \y \|_2$$
  with equality if and only if $\x = \alpha \y$ for some $\alpha \in \mathbb R$.
\end{proposition}
\begin{definition}[Neighborhood/Open ball]
  Given $\delta > 0$, $\bar\x\in \mathbb{R}^n$, the open ball $B_{\delta}(\bar{\x})=\{\x \in \mathbb{R}^n\ | \|\x-\bar{\x}\|_2< \delta\}$
\end{definition}
\begin{definition}[Sequence Convergence in $\mathbb R^n$]
  We say that a sequence $\{\x_k\} \subseteq \mathbb R^n$ converges to $\x^* \in \mathbb R^n$ and write $$\lim_{k \to \infty} \x_k = x^*$$ if for every $\epsilon > 0$ there exists $N \in \mathbb N$ such that for every $k \geq N$, $\| \x_k - \x^* \| < \epsilon$.
\end{definition}
\begin{definition}[Limit Point]
  If $\{\x_k\}$ has a subsequence that converges to $\x^*$, then $\x^*$ is called a limit point of $\{\x_k\}$.

  Given a set $E \subseteq \mathbb R^n$, if there exists a sequence $\{\x_k\} \subseteq E$ that converges to $\x^*$, then $\x^*$ is called a limit point of $E$.
\end{definition}
\begin{definition}[Closed Set]
  A set $E \subseteq \mathbb R^n$ is closed if it contains all of its limit points.

  That is, for every sequence $\{\x_k\} \subseteq E$ with $$\lim_{k \rightarrow \infty} \x_k = \x^*$$ if $\x^* \in E$, then $E$ is closed.
\end{definition}
\begin{definition}[Interior Point]
  A point $\x \in E$ is an interior point of $E$ is there is a neighborhood/open ball of $\x$ that is contained in $E$.
\end{definition}
\begin{definition}[Open Set]
  A set $E$ is open if all of its elements are interior points.

  So, for any point you pick in $E$, you can find a small neighborhood around that point which is entirely contained in $E$ (there are no boundary points in $E$).
\end{definition}
\begin{theorem}[Properties of Open/Closed Sets]
  The followings hold:
  \begin{itemize}
    \item A set is closed (open) if its complement, $\mathbb R^n \setminus E$, is open (closed).
    \item Union of finitely many closed sets is closed
    \item Intersection of (finitely or infinitely many) closed sets is closed.
    \item Intersection of finitely many open sets is open
    \item Union of (finitely or infinitely many) open sets is open.
  \end{itemize}
\end{theorem}
\begin{definition}[Bounded]
  A set $E \subset \mathbb R^n$ is bounded if it can be contained in a ball of finite radius. That is, there exists a neighborhood, $B_\delta(\x)$, such that $E \subseteq B_\delta(\x)$.
\end{definition}
\begin{definition}[Compact]
  A set $E \subset \mathbb R^n$ is compact if it is closed and bounded.
\end{definition}
\begin{definition}[Lipschitz Continuous/Contraction]
  Let $E \subseteq \mathbb R^n$. Let $f: E \to \mathbb R^m$ be a function. We say $f$ is Lipschitz continuous on $E$ if there exists a constant $L > 0$ such that for all $\x, \y \in E$,
  $$\| f(\x) - f(\y) \| \leq L \| \x - \y \|$$
\end{definition}
\begin{theorem}[Continuity of Functions]
  Let $E \subseteq \mathbb R^n, f,g: E \to \mathbb R^m$ and $\alpha \in \mathbb R$. If $f$ and $g$ are continuous at $\x_0$, then
  \begin{enumerate}
    \item $f + g, fg, \alpha f$ are continuous at $x_0$
    \item $\frac{f}{g}$ is continuous at $\x_0$ provided that $g(\x_0) \neq 0$.
  \end{enumerate}
\end{theorem}
\begin{definition}[Formal Definition of Derivative]
  The \textbf{derivative} of $f$ at $a$ is defined as $$f'(a) = \lim_{x \rightarrow a} \frac{f(x) - f(a)}{x-a}$$ if the limit exists.

  An alternate definition is
  $$f'(a) = \lim_{h \rightarrow 0} \frac{f(a+h) - f(a)}{h}$$
\end{definition}
\begin{theorem}[Extreme Value Theorem (EVT)]
  If $E \subset \mathbb R^n$ is a compact set (closed and bounded) and $f: E \to \mathbb R$ is a continuous function, then $f$ attains a maximum and minimum on $E$. That is, there exists points $\x_{\text{min}}, \x_{\text{max}} \in E$ such that for all $\x \in E$,
  $$f(x) \leq f(\x_{\text{max}}) \quad \text{and} \quad f(\x_{\text{min}}) \leq f(\x)$$
\end{theorem}
\begin{theorem}[Continuously Differentiable]
  $f$ is continuously differentiable at $\x_0$ if all partial derivatives exist and are continuous in a neighborhood of $\x_0$. We say $f$ is continuously differentiable, $f \in C^1$, if its partial derivatives are continuous everywhere.

  \bigskip
  $f$ is twice differentiable on $E$ if $\nabla^2 f(x)$ exists for all $\x \in E$. If each entry of the Hessian $\nabla^2 f(\x)$ is continuous, we say $f$ is twice differentiable on $E$, $f \in C^2$.
\end{theorem}
\begin{theorem}[]
  If $f: E \to \mathbb R$ is twice differentiable, then the Hessian is symmetric.
\end{theorem}
\begin{theorem}[Mean Value Theorem (MVT)]
  Let $f: \mathbb R \to \mathbb R$ be continuous on the closed interval $[a,b]$ and differentiable on the open interval $(a,b)$. Then there exists $c \in (a,b)$ such that
  $$f(b) - f(a) = f'(c) (b - a)$$
\end{theorem}
\begin{lemma}[Directional Derivative]
  Let $f: \mathbb R^n \rightarrow \mathbb R, \bar x, d \in \mathbb R^n$ where $d$ is the direction. We define
  $$\phi (\epsilon) = f(\bar x + \epsilon d): \mathbb R \rightarrow \mathbb R$$ the value of the function $f$ at a point that is displaced from $\bar x$ by a distance of $\epsilon$ in the direction $d$.
  Then the \textbf{directional derivative}, denoted $f'(x; d)$ of $f$ at $x$ at the direction $d$ is
  $$f'(x; d) = \phi'(0) = \lim_{\epsilon \rightarrow 0} \frac{f(x + \epsilon d) - f(x)}{\epsilon} = \nabla f(x)^T d$$
\end{lemma}
\begin{definition}[Directional Derivative (Different notation)]
  The directional derivative of $f$ at a point $\bar x, \in \mathbb R^n$ in the direction $d$ is $$f'(\bar x; d) = \frac{d}{ds} f(\bar x + sd) \bigg|_{s = 0}$$
\end{definition}
\begin{theorem}[]
  If $f$ is differentiable at $\bar x$, then $$f'(\bar x; d) = \nabla f(\bar x)^T d$$
\end{theorem}
\begin{proof}[Proof]
  We only prove in the case where $\bar x = (a, b) \in \mathbb R^2$.
  \begin{align*}
      f'(\bar x; d) &= \frac{d}{ds} f(\bar x + sd) \bigg|_{s = 0} \\
      &= \frac{d}{ds} f(\underbrace{a + sd_1}_{x}, \underbrace{b + sd_2}_{y}) \bigg|_{s = 0} \\
      &= \left[\frac{\partial f}{\partial x} \frac{dx}{ds} + \frac{\partial f}{\partial y}\frac{dy}{ds}\right]_{s=0} \tag*{Chain rule}\\
      &= \left[\frac{\partial}{\partial x}f(a + sd_1, b + sd_2) \cdot d_1 + \frac{\partial}{\partial y}f(a + sd_1, b + sd_2) \cdot d_2\right]_{s=0} \\
      &= \frac{\partial f}{\partial x} (a,b) \cdot d_1 + \frac{\partial f}{\partial y}(a,b) \cdot d_2 \\
      &= \left(\frac{\partial f}{\partial x}(a,b), \frac{\partial f}{\partial y} (a,b)\right) \cdot (d_1, d_2) \\
      &= \nabla f(a,b) \cdot d
  \end{align*}
\end{proof}
\begin{problem}
  Let $f(x,y,z) = x^2z + y^3z^2 - xyz$ with $d = \begin{pmatrix}
      -1 \\ 0 \\ 3
  \end{pmatrix}$ Then the \textbf{directional derivative} in the direction $d$ is
  $$\nabla f(x,y,z)^T d = \begin{pmatrix}
      2xz - yz \\
      3y^2z^2 - xz \\
      x^2 + 2y^3z - xy
  \end{pmatrix}^{T} \begin{pmatrix}
      -1 \\ 0 \\ 3
  \end{pmatrix} = -2xz + yz + 3x^2 + 6y^3z - 3xy$$
\end{problem}
\begin{theorem}[Taylor's Theorem on the real line]
  Let $f: (a, b) \rightarrow \mathbb R$, and $\bar{x}, x \in (a,b)$. Then the Taylor's series centered at $\bar x$ (approximation near $\bar x$) is $$f(x) = f(\bar x) + f'(\bar x)(x - \bar x) + \frac{f''(z)}{2}(x - \bar x)^2$$
  where $z$ is between $x$ and $\bar x$ that gives the largest value of $f''(z)$. The term $\frac{f''(z)}{2}(x - \bar x)^2$ is the \textbf{error term}

  Equivalently,
  $$f(\bar x + \Delta x) = \underbrace{f(x) + f'(x) \Delta x}_{\text{Linear approximation}} + o(|\Delta x|)(\text{little O})$$
  This formula emphasizes its use in approximating changes in $f$ for small changes in $x$, denoted $\Delta x$. $o(|\Delta x|)$, the error term, means that the error goes to 0 faster than $|\Delta x|$ as $\Delta x$ goes to 0. Therefore, this is saying that the linear approximation becomes more and more accurate for smaller $\Delta x$.
\end{theorem}
\begin{theorem}[Multivariate Taylor]
  Consider a $C^2$-smooth function $f: U \rightarrow \mathbb R$ on an open set $U \subset \mathbb R^n$. If $\bar x$ and $x$ are such that the segment $[\bar x, x] := \{\bar x + t(x - \bar x): t \in [0,1]\}$ is contained in $U$, then the Taylor series expansion of $f$ centered around $\bar x$ is
  $$f(x) = f(\bar x) + \langle \nabla f(\bar x), x - \bar x \rangle + \frac{1}{2} \langle \nabla^2 f(z) (x - \bar x), (x - \bar x) \rangle$$
  where $z$ is between $x$ and $\bar x$.
\end{theorem}
\begin{theorem}[Taylor's Theorem]
  Let $f: \mathbb R^n \to \mathbb R$ be continuously differentiable and let $\textbf{d} \in \mathbb R^n$ be a direction vector. Then
  $$f(\x + \textbf{d}) = f(\x) + \underbrace{\nabla f(\x + t\textbf{d})^T \textbf{d}}_{\text{directional derivative}}$$ for some $t \in (0,1)$. Moreover, if $f$ is twice differentiable, then
  $$f(\x + \textbf{d}) = f(\x) + \nabla f(\x)^T\textbf{d} + \frac{1}{2} \underbrace{\textbf{d}^T \nabla^2 f(\x + t\textbf{d}) \textbf{d}}_{\text{second order directional derivative}}$$ for some $t \in (0,1)$.

  \bigskip
  This theorem provides a way to approximate the value of the function $f$ at a point $\x + \textbf{d}$ based on the value and derivatives of $f$ at or near the point $\x$.
\end{theorem}
\begin{theorem}[Taylor's Thoerem (alternative)]
  Let $f: \mathbb R^n \to \mathbb R$ be continuously differentiable at $\x^*$. Then $\forall x \in \mathbb R^n$
  $$f(\x) = \underbrace{f(\x^*) + \nabla f(\x^*)^T (\x - \x^*)}_{\text{linear approximation of $f$ at $\x^*$}} + o(\| \x - \x^* \|)$$
  where $o(\| \x - \x^* \|)$ is the error term that goes to 0 faster than $\| \x - \x^* \|$ as $\x \to \x^*$.

  Let $f: \mathbb R^n \to \mathbb R$ be twice continuously differentiable at $\x^*$. Then $\forall x \in \mathbb R^n$
  $$f(\x) = \underbrace{f(\x^*) + \nabla f(\x^*)^T (\x - \x^*) + \frac{1}{2} (\x - \x^*)^T \nabla^2 f(\x^*) (\x - \x^*)}_{\text{quadratic approximation of $f$ at $\x^*$}} + o(\| \x - \x^* \|^2)$$
  where $o(\| \x - \x^* \|^2)$ is the error term that goes to 0 faster than $\| \x - \x^* \|^2$ as $\x \to \x^*$.
\end{theorem}
\begin{problem}[Lagrange Multiplier Example]
  Maximize $f(x,y) = x^2 y$ subject to $g(x,y) = x^2 + y^2 = 1$. By using Lagrange multipliers, we know that the maximizer, $(x^*, y^*)$ satisfies $$\nabla f(x^*, y^*) = \lambda \nabla g(x^*, y^*)$$ We have 
  \begin{align*}
    \nabla g(x,y) &= \begin{bmatrix} 2x \\ 2y \end{bmatrix} \\
    \nabla f(x,y) &= \begin{bmatrix} 2xy \\ x^2 \end{bmatrix}
  \end{align*} Then, using Lagrange multipliers, we solve
  $$\begin{bmatrix} 2xy \\ x^{2} \end{bmatrix} = \lambda \begin{bmatrix} 2x \\ 2y \end{bmatrix}$$ Since we also have the constraint $x^2 + y^2 = 1$, we solve the system of equations
  \begin{align*}
    2xy &= 2\lambda x \\
    x^2 &= 2\lambda y \\
    x^2 + y^2 &= 1
  \end{align*}
  Solving it, we get $$(x, y) = \left(\pm \sqrt{\frac{2}{3}}, \pm \sqrt{\frac{1}{3}}\right), \quad \lambda = y$$
  Testing each point, we get that $\left(\sqrt{\frac{2}{3}}, \sqrt{\frac{1}{3}}\right)$ is the maximizer of $f$.
\end{problem}

\subsection{Linear Algebra}
\begin{definition}[Characteristic Polynomial]
  The characteristic polynomial of $A \in \mathbb R^{n \times n}$ is 
  $$p(\lambda) = \det(A - \lambda I)$$ The degree of $p(\lambda)$ is $n$, and the leading term is $(-1)^n \lambda^n$.

  The eigenvalues of $A$ are the roots of the characteristic polynomial.
\end{definition}
\begin{definition}[Eigenvectors and Eigenvalues]
  $0 \neq v \in \mathbb R^n$ is an \textbf{eigenvector} of $A$ if there exists $\lambda \in \mathbb R$ such that $Av = \lambda v$. The number $\lambda$ is called an \textbf{eigenvalue} of $A$.
\end{definition}
\begin{theorem}[Finding Eigenvectors and Eigenvalues]
  Let $A$ be a matrix.
  \begin{enumerate}
    \item Set up the characteristic equation. We find $$\det(A - \lambda I) = 0$$
    \item Solve for $\lambda$. These are the eigenvalues.
    \item Plug eigenvalues $\lambda_1,\ldots, \lambda_n$ into $(A - \lambda I)v = 0$ and solve for $v$. These are the eigenvectors.
  \end{enumerate}
\end{theorem}
\begin{definition}[Quadratic Form]
  Let $A$ be a symmetric matrix and $x = \begin{pmatrix}
      x_1 \\ \vdots \\ x_n
  \end{pmatrix}$. The \textbf{quadratic form} $Q$ of the matrix $A$ is defined as $$Q = x^T Ax$$
\end{definition}

\begin{problem}[Example]
  Consider the matrix $A = \begin{bmatrix}
      5 & -5 \\
      -5 & 1
  \end{bmatrix}$. The quadratic form of $A$ is $$Q(x) = 5x^2_1 - 10x_1x_2 + x^2_2$$
\end{problem}
\begin{definition}[Classification of Quadratic Forms]
  Let $Q = x^TAx$ be a quadratic form of a matrix $A$. Then $A$ is
  \begin{enumerate}
      \item positive definite if $Q(x) > 0$ for all non-zero vectors $x$, and $Q(x) = 0$ if and only if $x = 0$. Or all eigenvalues of $A$ are positive. Denoted by $A \succ 0$.
      \item positive semidefinite if $Q(x) \geq 0$ for all vectors $x$, with $Q(x) = 0$ occurring for some non-zero vectors $x$. Or all eigenvalues of $A$ are non-negative. Denoted by $A \succeq 0$.
      \item negative definite if $Q(x) < 0$ for all non-zero vectors $x$, and $Q(x) = 0$ if and only if $x = 0$. Or all eigenvalues of $A$ are negative. Denoted by $A \prec 0$.
      \item negative semidefinite if $Q(x) \leq 0$ for all vectors $x$, with $Q(x) = 0$ occuring for some non-zero vectors $x$. Or all eigenvalues of $A$ are non-negative. Denoted by $A \preceq 0$.
      \item indefinite if $Q(x)$ can be positive or negative. Or there are positive and negative eigenvalues for $A$.
  \end{enumerate}
\end{definition}
\begin{theorem}[Orthogonal Spectral Decomposition]
  Let $A \in \mathbb S^n$. Then $A$ has an \textbf{orthogonal spectral decomposition}
  $$A = \sum_i \lambda_i u_i u_i^T = UDU^T$$ where $U$ is orthogonal with the orthogonal eigenvectors $u_i$ as columns and $D$ is a diagonal matrix with real eigenvalues on the diagonal.
\end{theorem}
\begin{proposition}
  Let $A \in \mathbb S^n$. The following are equivalent (Positive definite):
  \begin{enumerate}
    \item $A \succ 0$.
    \item All the eigenvalues of $A$ are in $\mathbb R^n_{++}$, the interior of the nonnegative orthant.
    \item $A$ has a real symmetric positive definite square root, $A = SS, S \in \mathbb S^n_{++}$.
    \item $A$ has a lower triangular factorization, a Cholesky factorization, $A = LL^T$ and $L$ has positive diagonal elements.
    \item All principal minors of $A$ are positive.
    \item All leading principal minors of $A$ are positive.
  \end{enumerate}
  And the following are equivalent (Positive semidefinite):
  \begin{enumerate}
    \item $A \succeq 0$.
    \item All the eigenvalues of $A$ are in $\mathbb R^n_{+}$, the nonnegative orthant.
    \item $A$ has a real symmetric square root, $A = SS, S \in \mathbb S^n$.
    \item $A$ has a lower triangular factorization, a Cholesky factorization, $A = LL^T$.
    \item All principal minors of $A$ are nonnegative.
  \end{enumerate}
\end{proposition}

\begin{definition}[Principal Submatrices]
  Let $$A = \begin{bmatrix}
    1 & 1 & 2 & 7 \\
    1 & 1 & 4 & 6 \\
    2 & 4 & 7 & 8 \\
    7 & 6 & 8 & 1 \\
  \end{bmatrix}, \quad I = \{1,3\}, \quad A[I] = \begin{bmatrix}
    1 & 2 \\
    2 & 7 \\
  \end{bmatrix}$$
  Then $A[I]$ is a \textbf{principal submatrix} of $A$.
\end{definition}
\begin{definition}[Principal Minors]
  Let $A \in \mathbb S^n$, where $\mathbb S^n$ is the set of all symmetric $n \times n$ matrices.
\begin{enumerate}
  \item $\det(A[I])$ is called the \textbf{principal minor} of $A$.
  \item If $I = \{1,\ldots, k\}$ then $\det(A[I])$ is called the \textbf{leading principal minor} of $A$.
\end{enumerate}
\end{definition}
\begin{proposition}[Characterizing Positive Definiteness with Principal Minors]
  Let $A \in \mathbb S^n$. Then
  \begin{enumerate}
    \item $A \succeq 0 \iff \det(A[I]) \geq 0$ for all principal minors $\det(A[I])$.
    \item $A \succ 0 \iff \det(A[I]) > 0$ for all \textbf{leading} principal minors $\det(A[I])$.
  \end{enumerate}
\end{proposition}
\begin{theorem}[]
  Let $A \in \mathbb R^{n \times n}$ be symmetric. Then the following are equivalent:
  \begin{enumerate}
    \item $A$ is positive semidefinite (definite).
    \item All eigenvalues of $A$ are nonnegative (positive).
    \item $A$ can be factored as $A = BB^T$ where $B$ is an $n \times p$ matrix for some $p$. (Cholesky factorization)
  \end{enumerate}
\end{theorem}
\begin{definition}[Diagonally Dominant]
  A matrix $A$ is diagonally dominant if for every $i$, $$|a_{ii}| \geq \sum_{j \neq i} |a_{ij}|$$
  So each diagonal element is greater than or equal to the sum of the absolute values of the other elements in the same row.

  It is called strictly diagonally dominant if for every $i$, $$|a_{ii}| > \sum_{j \neq i} |a_{ij}|$$
\end{definition}
\begin{proposition}
  Let $A \in \mathbb R^{n \times n}$ be a symmetric, diagonally dominant matrix whose diagonal entries are nonnegative, then $A$ is positive semidefinite.

  \bigskip
  Let $A \in \mathbb R^{n \times n}$ be a symmetric, strictly diagonally dominant matrix whose diagonal entries are positive, then $A$ is positive definite.

  \bigskip
  Note that the converse is not true.
\end{proposition}
\begin{definition}[Four Fundamental Subspaces]
  Let $A$ be a $m \times n$ matrix.
  \begin{itemize}
    \item The range space of $A$ is defined as $\text{Range}(A) = \{Ax: x \in \mathbb R^n\} \subseteq \mathbb R^m$.
    \item The range space of $A^T$ is defined as $\text{Range}(A^T) = \{A^T y: y \in \mathbb R^m\} \subseteq \mathbb R^n$.
    \item The null space of $A$ is defined as $\text{Null}(A) = \{x \in \mathbb R^n: Ax = 0\} \subseteq \mathbb R^n$.
    \item The null space of $A^T$ is defined as $\text{Null}(A^T) = \{y \in \mathbb R^m: A^T y = 0\} \subseteq \mathbb R^m$.
  \end{itemize}
\end{definition}
\begin{theorem}[Rank Nullity Theorem]
  Let $A$ be an $m \times n$ matrix. Then
  $$\text{rank}(A) + \dim(\text{Null}(A)) = n$$
  Recall that rank of a matrix is the number of pivots in the reduced row echelon form of the matrix. The dimension of a subspace is the number of linearly independent vectors that span the subspace.
\end{theorem}